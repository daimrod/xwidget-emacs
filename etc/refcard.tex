% Reference Card for GNU Emacs version 20 on Unix systems
%**start of header
\newcount\columnsperpage

% This file can be printed with 1, 2, or 3 columns per page (see below).
% Specify how many you want here.  Nothing else needs to be changed.

\columnsperpage=1

% Copyright (c) 1987, 1993, 1996, 1997 Free Software Foundation, Inc.

% This file is part of GNU Emacs.

% GNU Emacs is free software; you can redistribute it and/or modify
% it under the terms of the GNU General Public License as published by
% the Free Software Foundation; either version 2, or (at your option)
% any later version.

% GNU Emacs is distributed in the hope that it will be useful,
% but WITHOUT ANY WARRANTY; without even the implied warranty of
% MERCHANTABILITY or FITNESS FOR A PARTICULAR PURPOSE.  See the
% GNU General Public License for more details.

% You should have received a copy of the GNU General Public License
% along with GNU Emacs; see the file COPYING.  If not, write to
% the Free Software Foundation, Inc., 59 Temple Place - Suite 330,
% Boston, MA 02111-1307, USA.

% This file is intended to be processed by plain TeX (TeX82).
%
% The final reference card has six columns, three on each side.
% This file can be used to produce it in any of three ways:
% 1 column per page
%    produces six separate pages, each of which needs to be reduced to 80%.
%    This gives the best resolution.
% 2 columns per page
%    produces three already-reduced pages.
%    You will still need to cut and paste.
% 3 columns per page
%    produces two pages which must be printed sideways to make a
%    ready-to-use 8.5 x 11 inch reference card.
%    For this you need a dvi device driver that can print sideways.
% Which mode to use is controlled by setting \columnsperpage above.
%
% Author:
%  Stephen Gildea
%  Internet: gildea@stop.mail-abuse.org
%
% Thanks to Paul Rubin, Bob Chassell, Len Tower, and Richard Mlynarik
% for their many good ideas.

% If there were room, it would be nice to see a section on Dired.

\def\versionnumber{2.2}
\def\year{1997}

\def\shortcopyrightnotice{\vskip 1ex plus 2 fill
  \centerline{\small \copyright\ \year\ Free Software Foundation, Inc.
  Permissions on back.  v\versionnumber}}

\def\copyrightnotice{
\vskip 1ex plus 2 fill\begingroup\small
\centerline{Copyright \copyright\ \year\ Free Software Foundation, Inc.}
\centerline{v\versionnumber{} for GNU Emacs version 20, June \year}
\centerline{designed by Stephen Gildea}

Permission is granted to make and distribute copies of
this card provided the copyright notice and this permission notice
are preserved on all copies.

For copies of the GNU Emacs manual, write to the Free Software
Foundation, Inc., 59 Temple Place, Suite 330, Boston, MA  02111-1307  USA

\endgroup}

% make \bye not \outer so that the \def\bye in the \else clause below
% can be scanned without complaint.
\def\bye{\par\vfill\supereject\end}

\newdimen\intercolumnskip	%horizontal space between columns
\newbox\columna			%boxes to hold columns already built
\newbox\columnb

\def\ncolumns{\the\columnsperpage}

\message{[\ncolumns\space 
  column\if 1\ncolumns\else s\fi\space per page]}

\def\scaledmag#1{ scaled \magstep #1}

% This multi-way format was designed by Stephen Gildea October 1986.
% Note that the 1-column format is fontfamily-independent.
\if 1\ncolumns			%one-column format uses normal size
  \hsize 4in
  \vsize 10in
  \voffset -.7in
  \font\titlefont=\fontname\tenbf \scaledmag3
  \font\headingfont=\fontname\tenbf \scaledmag2
  \font\smallfont=\fontname\sevenrm
  \font\smallsy=\fontname\sevensy

  \footline{\hss\folio}
  \def\makefootline{\baselineskip10pt\hsize6.5in\line{\the\footline}}
\else				%2 or 3 columns uses prereduced size
  \hsize 3.2in
  \vsize 7.95in
  \hoffset -.75in
  \voffset -.745in
  \font\titlefont=cmbx10 \scaledmag2
  \font\headingfont=cmbx10 \scaledmag1
  \font\smallfont=cmr6
  \font\smallsy=cmsy6
  \font\eightrm=cmr8
  \font\eightbf=cmbx8
  \font\eightit=cmti8
  \font\eighttt=cmtt8
  \font\eightmi=cmmi8
  \font\eightsy=cmsy8
  \textfont0=\eightrm
  \textfont1=\eightmi
  \textfont2=\eightsy
  \def\rm{\eightrm}
  \def\bf{\eightbf}
  \def\it{\eightit}
  \def\tt{\eighttt}
  \normalbaselineskip=.8\normalbaselineskip
  \normallineskip=.8\normallineskip
  \normallineskiplimit=.8\normallineskiplimit
  \normalbaselines\rm		%make definitions take effect

  \if 2\ncolumns
    \let\maxcolumn=b
    \footline{\hss\rm\folio\hss}
    \def\makefootline{\vskip 2in \hsize=6.86in\line{\the\footline}}
  \else \if 3\ncolumns
    \let\maxcolumn=c
    \nopagenumbers
  \else
    \errhelp{You must set \columnsperpage equal to 1, 2, or 3.}
    \errmessage{Illegal number of columns per page}
  \fi\fi

  \intercolumnskip=.46in
  \def\abc{a}
  \output={%			%see The TeXbook page 257
      % This next line is useful when designing the layout.
      %\immediate\write16{Column \folio\abc\space starts with \firstmark}
      \if \maxcolumn\abc \multicolumnformat \global\def\abc{a}
      \else\if a\abc
	\global\setbox\columna\columnbox \global\def\abc{b}
        %% in case we never use \columnb (two-column mode)
        \global\setbox\columnb\hbox to -\intercolumnskip{}
      \else
	\global\setbox\columnb\columnbox \global\def\abc{c}\fi\fi}
  \def\multicolumnformat{\shipout\vbox{\makeheadline
      \hbox{\box\columna\hskip\intercolumnskip
        \box\columnb\hskip\intercolumnskip\columnbox}
      \makefootline}\advancepageno}
  \def\columnbox{\leftline{\pagebody}}

  \def\bye{\par\vfill\supereject
    \if a\abc \else\null\vfill\eject\fi
    \if a\abc \else\null\vfill\eject\fi
    \end}  
\fi

% we won't be using math mode much, so redefine some of the characters
% we might want to talk about
\catcode`\^=12
\catcode`\_=12

\chardef\\=`\\
\chardef\{=`\{
\chardef\}=`\}

\hyphenation{mini-buf-fer}

\parindent 0pt
\parskip 1ex plus .5ex minus .5ex

\def\small{\smallfont\textfont2=\smallsy\baselineskip=.8\baselineskip}

% newcolumn - force a new column.  Use sparingly, probably only for
% the first column of a page, which should have a title anyway.
\outer\def\newcolumn{\vfill\eject}

% title - page title.  Argument is title text.
\outer\def\title#1{{\titlefont\centerline{#1}}\vskip 1ex plus .5ex}

% section - new major section.  Argument is section name.
\outer\def\section#1{\par\filbreak
  \vskip 3ex plus 2ex minus 2ex {\headingfont #1}\mark{#1}%
  \vskip 2ex plus 1ex minus 1.5ex}

\newdimen\keyindent

% beginindentedkeys...endindentedkeys - key definitions will be
% indented, but running text, typically used as headings to group
% definitions, will not.
\def\beginindentedkeys{\keyindent=1em}
\def\endindentedkeys{\keyindent=0em}
\endindentedkeys

% paralign - begin paragraph containing an alignment.
% If an \halign is entered while in vertical mode, a parskip is never
% inserted.  Using \paralign instead of \halign solves this problem.
\def\paralign{\vskip\parskip\halign}

% \<...> - surrounds a variable name in a code example
\def\<#1>{{\it #1\/}}

% kbd - argument is characters typed literally.  Like the Texinfo command.
\def\kbd#1{{\tt#1}\null}	%\null so not an abbrev even if period follows

% beginexample...endexample - surrounds literal text, such a code example.
% typeset in a typewriter font with line breaks preserved
\def\beginexample{\par\leavevmode\begingroup
  \obeylines\obeyspaces\parskip0pt\tt}
{\obeyspaces\global\let =\ }
\def\endexample{\endgroup}

% key - definition of a key.
% \key{description of key}{key-name}
% prints the description left-justified, and the key-name in a \kbd
% form near the right margin.
\def\key#1#2{\leavevmode\hbox to \hsize{\vtop
  {\hsize=.75\hsize\rightskip=1em
  \hskip\keyindent\relax#1}\kbd{#2}\hfil}}

\newbox\metaxbox
\setbox\metaxbox\hbox{\kbd{M-x }}
\newdimen\metaxwidth
\metaxwidth=\wd\metaxbox

% metax - definition of a M-x command.
% \metax{description of command}{M-x command-name}
% Tries to justify the beginning of the command name at the same place
% as \key starts the key name.  (The "M-x " sticks out to the left.)
\def\metax#1#2{\leavevmode\hbox to \hsize{\hbox to .75\hsize
  {\hskip\keyindent\relax#1\hfil}%
  \hskip -\metaxwidth minus 1fil
  \kbd{#2}\hfil}}

% threecol - like "key" but with two key names.
% for example, one for doing the action backward, and one for forward.
\def\threecol#1#2#3{\hskip\keyindent\relax#1\hfil&\kbd{#2}\hfil\quad
  &\kbd{#3}\hfil\quad\cr}

%**end of header


\title{GNU Emacs Reference Card}

\centerline{(for version 20)}

\section{Starting Emacs}

To enter GNU Emacs 20, just type its name: \kbd{emacs}

To read in a file to edit, see Files, below.

\section{Leaving Emacs}

\key{suspend Emacs (or iconify it under X)}{C-z}
\key{exit Emacs permanently}{C-x C-c}

\section{Files}

\key{{\bf read} a file into Emacs}{C-x C-f}
\key{{\bf save} a file back to disk}{C-x C-s}
\key{save {\bf all} files}{C-x s}
\key{{\bf insert} contents of another file into this buffer}{C-x i}
\key{replace this file with the file you really want}{C-x C-v}
\key{write buffer to a specified file}{C-x C-w}
\key{version control checkin/checkout}{C-x C-q}

\section{Getting Help}

The help system is simple.  Type \kbd{C-h} (or \kbd{F1}) and follow
the directions.  If you are a first-time user, type \kbd{C-h t} for a
{\bf tutorial}.

\key{remove help window}{C-x 1}
\key{scroll help window}{C-M-v}

\key{apropos: show commands matching a string}{C-h a}
\key{show the function a key runs}{C-h c}
\key{describe a function}{C-h f}
\key{get mode-specific information}{C-h m}

\section{Error Recovery}

\key{{\bf abort} partially typed or executing command}{C-g}
\metax{{\bf recover} a file lost by a system crash}{M-x recover-file}
\key{{\bf undo} an unwanted change}{C-x u {\rm or} C-_}
\metax{restore a buffer to its original contents}{M-x revert-buffer}
\key{redraw garbaged screen}{C-l}

\section{Incremental Search}

\key{search forward}{C-s}
\key{search backward}{C-r}
\key{regular expression search}{C-M-s}
\key{reverse regular expression search}{C-M-r}

\key{select previous search string}{M-p}
\key{select next later search string}{M-n}
\key{exit incremental search}{RET}
\key{undo effect of last character}{DEL}
\key{abort current search}{C-g}

Use \kbd{C-s} or \kbd{C-r} again to repeat the search in either direction.
If Emacs is still searching, \kbd{C-g} cancels only the part not done.

\shortcopyrightnotice

\section{Motion}

\paralign to \hsize{#\tabskip=10pt plus 1 fil&#\tabskip=0pt&#\cr
\threecol{{\bf entity to move over}}{{\bf backward}}{{\bf forward}}
\threecol{character}{C-b}{C-f}
\threecol{word}{M-b}{M-f}
\threecol{line}{C-p}{C-n}
\threecol{go to line beginning (or end)}{C-a}{C-e}
\threecol{sentence}{M-a}{M-e}
\threecol{paragraph}{M-\{}{M-\}}
\threecol{page}{C-x [}{C-x ]}
\threecol{sexp}{C-M-b}{C-M-f}
\threecol{function}{C-M-a}{C-M-e}
\threecol{go to buffer beginning (or end)}{M-<}{M->}
}

\key{scroll to next screen}{C-v}
\key{scroll to previous screen}{M-v}
\key{scroll left}{C-x <}
\key{scroll right}{C-x >}
\key{scroll current line to center of screen}{C-u C-l}

\section{Killing and Deleting}

\paralign to \hsize{#\tabskip=10pt plus 1 fil&#\tabskip=0pt&#\cr
\threecol{{\bf entity to kill}}{{\bf backward}}{{\bf forward}}
\threecol{character (delete, not kill)}{DEL}{C-d}
\threecol{word}{M-DEL}{M-d}
\threecol{line (to end of)}{M-0 C-k}{C-k}
\threecol{sentence}{C-x DEL}{M-k}
\threecol{sexp}{M-- C-M-k}{C-M-k}
}

\key{kill {\bf region}}{C-w}
\key{copy region to kill ring}{M-w}
\key{kill through next occurrence of {\it char}}{M-z {\it char}}

\key{yank back last thing killed}{C-y}
\key{replace last yank with previous kill}{M-y}

\section{Marking}

\key{set mark here}{C-@ {\rm or} C-SPC}
\key{exchange point and mark}{C-x C-x}

\key{set mark {\it arg\/} {\bf words} away}{M-@}
\key{mark {\bf paragraph}}{M-h}
\key{mark {\bf page}}{C-x C-p}
\key{mark {\bf sexp}}{C-M-@}
\key{mark {\bf function}}{C-M-h}
\key{mark entire {\bf buffer}}{C-x h}

\section{Query Replace}

\key{interactively replace a text string}{M-\%}
\metax{using regular expressions}{M-x query-replace-regexp}

Valid responses in query-replace mode are

\key{{\bf replace} this one, go on to next}{SPC}
\key{replace this one, don't move}{,}
\key{{\bf skip} to next without replacing}{DEL}
\key{replace all remaining matches}{!}
\key{{\bf back up} to the previous match}{^}
\key{{\bf exit} query-replace}{RET}
\key{enter recursive edit (\kbd{C-M-c} to exit)}{C-r}

\section{Multiple Windows}

When two commands are shown, the second is for ``other frame.''

\key{delete all other windows}{C-x 1}

{\setbox0=\hbox{\kbd{0}}\advance\hsize by 0\wd0
\paralign to \hsize{#\tabskip=10pt plus 1 fil&#\tabskip=0pt&#\cr
\threecol{split window, above and below}{C-x 2\ \ \ \ }{C-x 5 2}
\threecol{delete this window}{C-x 0\ \ \ \ }{C-x 5 0}
}}
\key{split window, side by side}{C-x 3}

\key{scroll other window}{C-M-v}

{\setbox0=\hbox{\kbd{0}}\advance\hsize by 2\wd0
\paralign to \hsize{#\tabskip=10pt plus 1 fil&#\tabskip=0pt&#\cr
\threecol{switch cursor to another window}{C-x o}{C-x 5 o}

\threecol{select buffer in other window}{C-x 4 b}{C-x 5 b}
\threecol{display buffer in other window}{C-x 4 C-o}{C-x 5 C-o}
\threecol{find file in other window}{C-x 4 f}{C-x 5 f}
\threecol{find file read-only in other window}{C-x 4 r}{C-x 5 r}
\threecol{run Dired in other window}{C-x 4 d}{C-x 5 d}
\threecol{find tag in other window}{C-x 4 .}{C-x 5 .}
}}

\key{grow window taller}{C-x ^}
\key{shrink window narrower}{C-x \{}
\key{grow window wider}{C-x \}}

\section{Formatting}

\key{indent current {\bf line} (mode-dependent)}{TAB}
\key{indent {\bf region} (mode-dependent)}{C-M-\\}
\key{indent {\bf sexp} (mode-dependent)}{C-M-q}
\key{indent region rigidly {\it arg\/} columns}{C-x TAB}

\key{insert newline after point}{C-o}
\key{move rest of line vertically down}{C-M-o}
\key{delete blank lines around point}{C-x C-o}
\key{join line with previous (with arg, next)}{M-^}
\key{delete all white space around point}{M-\\}
\key{put exactly one space at point}{M-SPC}

\key{fill paragraph}{M-q}
\key{set fill column}{C-x f}
\key{set prefix each line starts with}{C-x .}

\key{set face}{M-g}

\section{Case Change}

\key{uppercase word}{M-u}
\key{lowercase word}{M-l}
\key{capitalize word}{M-c}

\key{uppercase region}{C-x C-u}
\key{lowercase region}{C-x C-l}

\section{The Minibuffer}

The following keys are defined in the minibuffer.

\key{complete as much as possible}{TAB}
\key{complete up to one word}{SPC}
\key{complete and execute}{RET}
\key{show possible completions}{?}
\key{fetch previous minibuffer input}{M-p}
\key{fetch later minibuffer input or default}{M-n}
\key{regexp search backward through history}{M-r}
\key{regexp search forward through history}{M-s}
\key{abort command}{C-g}

Type \kbd{C-x ESC ESC} to edit and repeat the last command that used the
minibuffer.  Type \kbd{F10} to activate the menu bar using the minibuffer.

\newcolumn
\title{GNU Emacs Reference Card}

\section{Buffers}

\key{select another buffer}{C-x b}
\key{list all buffers}{C-x C-b}
\key{kill a buffer}{C-x k}

\section{Transposing}

\key{transpose {\bf characters}}{C-t}
\key{transpose {\bf words}}{M-t}
\key{transpose {\bf lines}}{C-x C-t}
\key{transpose {\bf sexps}}{C-M-t}

\section{Spelling Check}

\key{check spelling of current word}{M-\$}
\metax{check spelling of all words in region}{M-x ispell-region}
\metax{check spelling of entire buffer}{M-x ispell-buffer}

\section{Tags}

\key{find a tag (a definition)}{M-.}
\key{find next occurrence of tag}{C-u M-.}
\metax{specify a new tags file}{M-x visit-tags-table}

\metax{regexp search on all files in tags table}{M-x tags-search}
\metax{run query-replace on all the files}{M-x tags-query-replace}
\key{continue last tags search or query-replace}{M-,}

\section{Shells}

\key{execute a shell command}{M-!}
\key{run a shell command on the region}{M-|}
\key{filter region through a shell command}{C-u M-|}
\key{start a shell in window \kbd{*shell*}}{M-x shell}

\section{Rectangles}

\key{copy rectangle to register}{C-x r r}
\key{kill rectangle}{C-x r k}
\key{yank rectangle}{C-x r y}
\key{open rectangle, shifting text right}{C-x r o}
\key{blank out rectangle}{C-x r c}
\key{prefix each line with a string}{C-x r t}

\section{Abbrevs}

\key{add global abbrev}{C-x a g}
\key{add mode-local abbrev}{C-x a l}
\key{add global expansion for this abbrev}{C-x a i g}
\key{add mode-local expansion for this abbrev}{C-x a i l}
\key{explicitly expand abbrev}{C-x a e}

\key{expand previous word dynamically}{M-/}

\section{Regular Expressions}

\key{any single character except a newline}{. {\rm(dot)}}
\key{zero or more repeats}{*}
\key{one or more repeats}{+}
\key{zero or one repeat}{?}
\key{quote regular expression special character {\it c\/}}{\\{\it c}}
\key{alternative (``or'')}{\\|}
\key{grouping}{\\( {\rm$\ldots$} \\)}
\key{same text as {\it n\/}th group}{\\{\it n}}
\key{at word break}{\\b}
\key{not at word break}{\\B}

\paralign to \hsize{#\tabskip=10pt plus 1 fil&#\tabskip=0pt&#\cr
\threecol{{\bf entity}}{{\bf match start}}{{\bf match end}}
\threecol{line}{^}{\$}
\threecol{word}{\\<}{\\>}
\threecol{buffer}{\\`}{\\'}

\threecol{{\bf class of characters}}{{\bf match these}}{{\bf match others}}
\threecol{explicit set}{[ {\rm$\ldots$} ]}{[^ {\rm$\ldots$} ]}
\threecol{word-syntax character}{\\w}{\\W}
\threecol{character with syntax {\it c}}{\\s{\it c}}{\\S{\it c}}
}

\section{International Character Sets}

\metax{specify principal language}{M-x set-language-environment}
\metax{show all input methods}{M-x list-input-methods}
\key{enable or disable input method}{C-\\}
\key{set coding system for next command}{C-x RET c}
\metax{show all coding systems}{M-x list-coding-systems}
\metax{choose preferred coding system}{M-x prefer-coding-system}

\section{Info}

\key{enter the Info documentation reader}{C-h i}
\key{find specified function or variable in Info}{C-h C-i}
\beginindentedkeys

Moving within a node:

\key{scroll forward}{SPC}
\key{scroll reverse}{DEL}
\key{beginning of node}{. {\rm (dot)}}

Moving between nodes:

\key{{\bf next} node}{n}
\key{{\bf previous} node}{p}
\key{move {\bf up}}{u}
\key{select menu item by name}{m}
\key{select {\it n\/}th menu item by number (1--9)}{{\it n}}
\key{follow cross reference  (return with \kbd{l})}{f}
\key{return to last node you saw}{l}
\key{return to directory node}{d}
\key{go to any node by name}{g}

Other:

\key{run Info {\bf tutorial}}{h}
\key{{\bf quit} Info}{q}
\key{search nodes for regexp}{M-s}

\endindentedkeys

\section{Registers}

\key{save region in register}{C-x r s}
\key{insert register contents into buffer}{C-x r i}

\key{save value of point in register}{C-x r SPC}
\key{jump to point saved in register}{C-x r j}

\section{Keyboard Macros}

\key{{\bf start} defining a keyboard macro}{C-x (}
\key{{\bf end} keyboard macro definition}{C-x )}
\key{{\bf execute} last-defined keyboard macro}{C-x e}
\key{append to last keyboard macro}{C-u C-x (}
\metax{name last keyboard macro}{M-x name-last-kbd-macro}
\metax{insert Lisp definition in buffer}{M-x insert-kbd-macro}

\section{Commands Dealing with Emacs Lisp}

\key{eval {\bf sexp} before point}{C-x C-e}
\key{eval current {\bf defun}}{C-M-x}
\metax{eval {\bf region}}{M-x eval-region}
\key{read and eval minibuffer}{M-:}
\metax{load from standard system directory}{M-x load-library}

\section{Simple Customization}

\metax{customize variables and faces}{M-x customize}

% The intended audience here is the person who wants to make simple
% customizations and knows Lisp syntax.

Making global key bindings in Emacs Lisp (examples):

\beginexample%
(global-set-key "\\C-cg" 'goto-line)
(global-set-key "\\M-\#" 'query-replace-regexp)
\endexample

\section{Writing Commands}

\beginexample%
(defun \<command-name> (\<args>)
  "\<documentation>" (interactive "\<template>")
  \<body>)
\endexample

An example:

\beginexample%
(defun this-line-to-top-of-window (line)
  "Reposition line point is on to top of window.
With ARG, put point on line ARG."
  (interactive "P")
  (recenter (if (null line)
                0
              (prefix-numeric-value line))))
\endexample

The \kbd{interactive} spec says how to read arguments interactively.
Type \kbd{C-h f interactive} for more details.

\copyrightnotice

\bye

% Local variables:
% compile-command: "tex refcard"
% End:
