% Reference Card for Org Mode 3.17
%
%**start of header
\newcount\columnsperpage

% This file can be printed with 1, 2, or 3 columns per page (see below).
% At the moment this card works quite nicely in 3 column format and
% currently takes 2 full pages.  It is thus suited to producing one 
% double-side page when printed.

% There are a couple of tweaks in the format of this card which make it work
% slightly better on A4 paper.  The changes due, I think, to Stephen Eglen,
% are commented below.  Revert the changes if you want letter sized paper.

% Try running something like
% tex org-mode-ref.tex; dvips -t landscape org-mode-ref.dvi
% to print the card

\columnsperpage=3

% Copyright (c) 2004 Philip Rooke, 2005 Free Software Foundation

% GNU Emacs is free software; you can redistribute it and/or modify
% it under the terms of the GNU General Public License as published by
% the Free Software Foundation; either version 2, or (at your option)
% any later version.

% GNU Emacs is distributed in the hope that it will be useful,
% but WITHOUT ANY WARRANTY; without even the implied warranty of
% MERCHANTABILITY or FITNESS FOR A PARTICULAR PURPOSE.  See the
% GNU General Public License for more details.

% You should have received a copy of the GNU General Public License
% along with GNU Emacs; see the file COPYING.  If not, write to
% the Free Software Foundation, Inc., 59 Temple Place - Suite 330,
% Boston, MA 02111-1307, USA.

% This file is intended to be processed by plain TeX (TeX82).
%
% The final reference card has six columns, three on each side.
% This file can be used to produce it in any of three ways:
% 1 column per page
%    produces six separate pages, each of which needs to be reduced to 80%.
%    This gives the best resolution.
% 2 columns per page
%    produces three already-reduced pages.
%    You will still need to cut and paste.
% 3 columns per page
%    produces two pages which must be printed sideways to make a
%    ready-to-use 8.5 x 11 inch reference card.
%    For this you need a dvi device driver that can print sideways.
% Which mode to use is controlled by setting \columnsperpage above.
%
% Author:
%  Stephen Gildea
%  Internet: gildea@stop.mail-abuse.org
%
% Thanks to Paul Rubin, Bob Chassell, Len Tower, and Richard Mlynarik
% for their many good ideas.

\def\orgversionnumber{3.17}
\def\year{2005}

\def\shortcopyrightnotice{\vskip 1ex plus 2 fill
  \centerline{\small \copyright\ 2004, 2005\ Free Software Foundation, Inc.
  Permissions on back.  v\orgversionnumber}}

\def\copyrightnotice{
\vskip 1ex plus 100 fill\begingroup\small
\centerline{Copyright \copyright\ \year\ Free Software Foundation, Inc.}
\centerline{v\orgversionnumber{} for Org-Mode \orgversionnumber{}, \year}
\centerline{Author: Philip Rooke}
\centerline{based on refcard design and format by Stephen Gildea}

Permission is granted to make and distribute copies of
this card provided the copyright notice and this permission notice
are preserved on all copies.

\endgroup}

% make \bye not \outer so that the \def\bye in the \else clause below
% can be scanned without complaint.
\def\bye{\par\vfill\supereject\end}

\newdimen\intercolumnskip       %horizontal space between columns
\newbox\columna                 %boxes to hold columns already built
\newbox\columnb

\def\ncolumns{\the\columnsperpage}

\message{[\ncolumns\space 
  column\if 1\ncolumns\else s\fi\space per page]}

\def\scaledmag#1{ scaled \magstep #1}

% This multi-way format was designed by Stephen Gildea October 1986.
% Note that the 1-column format is fontfamily-independent.
\if 1\ncolumns                  %one-column format uses normal size
  \hsize 4in
  \vsize 10in
  \voffset -.7in
  \font\titlefont=\fontname\tenbf \scaledmag3
  \font\headingfont=\fontname\tenbf \scaledmag2
  \font\smallfont=\fontname\sevenrm
  \font\smallsy=\fontname\sevensy

  \footline{\hss\folio}
  \def\makefootline{\baselineskip10pt\hsize6.5in\line{\the\footline}}
\else                           %2 or 3 columns uses prereduced size
  \hsize 3.2in

% FIXME - Try to make things more A4 friendly
%  \vsize 7.95in
  \vsize 7.65in
%  \hoffset -.75in
  \hoffset -.25in
  \voffset -.745in
  \font\titlefont=cmbx10 \scaledmag2
  \font\headingfont=cmbx10 \scaledmag1
  \font\smallfont=cmr6
  \font\smallsy=cmsy6
  \font\eightrm=cmr8
  \font\eightbf=cmbx8
  \font\eightit=cmti8
  \font\eighttt=cmtt8
  \font\eightmi=cmmi8
  \font\eightsy=cmsy8
  \textfont0=\eightrm
  \textfont1=\eightmi
  \textfont2=\eightsy
  \def\rm{\eightrm}
  \def\bf{\eightbf}
  \def\it{\eightit}
  \def\tt{\eighttt}

% FIXME - Try to make things more A4 friendly
%  \normalbaselineskip=.8\normalbaselineskip
  \normalbaselineskip=.7\normalbaselineskip
  \normallineskip=.8\normallineskip
  \normallineskiplimit=.8\normallineskiplimit
  \normalbaselines\rm           %make definitions take effect

  \if 2\ncolumns
    \let\maxcolumn=b
    \footline{\hss\rm\folio\hss}
    \def\makefootline{\vskip 2in \hsize=6.86in\line{\the\footline}}
  \else \if 3\ncolumns
    \let\maxcolumn=c
    \nopagenumbers
  \else
    \errhelp{You must set \columnsperpage equal to 1, 2, or 3.}
    \errmessage{Illegal number of columns per page}
  \fi\fi

  \intercolumnskip=.46in
  \def\abc{a}
  \output={%                    %see The TeXbook page 257
      % This next line is useful when designing the layout.
      %\immediate\write16{Column \folio\abc\space starts with \firstmark}
      \if \maxcolumn\abc \multicolumnformat \global\def\abc{a}
      \else\if a\abc
        \global\setbox\columna\columnbox \global\def\abc{b}
        %% in case we never use \columnb (two-column mode)
        \global\setbox\columnb\hbox to -\intercolumnskip{}
      \else
        \global\setbox\columnb\columnbox \global\def\abc{c}\fi\fi}
  \def\multicolumnformat{\shipout\vbox{\makeheadline
      \hbox{\box\columna\hskip\intercolumnskip
        \box\columnb\hskip\intercolumnskip\columnbox}
      \makefootline}\advancepageno}
  \def\columnbox{\leftline{\pagebody}}

  \def\bye{\par\vfill\supereject
    \if a\abc \else\null\vfill\eject\fi
    \if a\abc \else\null\vfill\eject\fi
    \end}  
\fi

% we won't be using math mode much, so redefine some of the characters
% we might want to talk about
%\catcode`\^=12
\catcode`\_=12

% we also need the tilde, for file names.
\catcode`\~=12

\chardef\\=`\\
\chardef\{=`\{
\chardef\}=`\}

\hyphenation{mini-buf-fer}

\parindent 0pt
\parskip 1ex plus .5ex minus .5ex

\def\small{\smallfont\textfont2=\smallsy\baselineskip=.8\baselineskip}

% newcolumn - force a new column.  Use sparingly, probably only for
% the first column of a page, which should have a title anyway.
\outer\def\newcolumn{\vfill\eject}

% title - page title.  Argument is title text.
\outer\def\title#1{{\titlefont\centerline{#1}}\vskip 1ex plus .5ex}

% section - new major section.  Argument is section name.
\outer\def\section#1{\par\filbreak
  \vskip 3ex plus 2ex minus 2ex {\headingfont #1}\mark{#1}%
  \vskip 2ex plus 1ex minus 1.5ex}

\newdimen\keyindent

% beginindentedkeys...endindentedkeys - key definitions will be
% indented, but running text, typically used as headings to group
% definitions, will not.
\def\beginindentedkeys{\keyindent=1em}
\def\endindentedkeys{\keyindent=0em}
\endindentedkeys

% paralign - begin paragraph containing an alignment.
% If an \halign is entered while in vertical mode, a parskip is never
% inserted.  Using \paralign instead of \halign solves this problem.
\def\paralign{\vskip\parskip\halign}

% \<...> - surrounds a variable name in a code example
\def\<#1>{{\it #1\/}}

% kbd - argument is characters typed literally.  Like the Texinfo command.
\def\kbd#1{{\tt#1}\null}        %\null so not an abbrev even if period follows

% beginexample...endexample - surrounds literal text, such a code example.
% typeset in a typewriter font with line breaks preserved
\def\beginexample{\par\leavevmode\begingroup
  \obeylines\obeyspaces\parskip0pt\tt}
{\obeyspaces\global\let =\ }
\def\endexample{\endgroup}

% key - definition of a key.
% \key{description of key}{key-name}
% prints the description left-justified, and the key-name in a \kbd
% form near the right margin.
\def\key#1#2{\leavevmode\hbox to \hsize{\vtop
  {\hsize=.75\hsize\rightskip=1em
  \hskip\keyindent\relax#1}\kbd{#2}\hfil}}

\newbox\metaxbox
\setbox\metaxbox\hbox{\kbd{M-x }}
\newdimen\metaxwidth
\metaxwidth=\wd\metaxbox

% metax - definition of a M-x command.
% \metax{description of command}{M-x command-name}
% Tries to justify the beginning of the command name at the same place
% as \key starts the key name.  (The "M-x " sticks out to the left.)
\def\metax#1#2{\leavevmode\hbox to \hsize{\hbox to .75\hsize
  {\hskip\keyindent\relax#1\hfil}%
  \hskip -\metaxwidth minus 1fil
  \kbd{#2}\hfil}}

% threecol - like "key" but with two key names.
% for example, one for doing the action backward, and one for forward.
\def\threecol#1#2#3{\hskip\keyindent\relax#1\hfil&\kbd{#2}\hfil\quad
  &\kbd{#3}\hfil\quad\cr}

%**end of header


\title{Org-Mode Reference Card (1/2)}

\centerline{(for version \orgversionnumber)}

\section{Getting Started}
%
Put the following in your \kbd{~/.emacs}$^1$
\vskip -1mm
\beginexample%
(autoload 'org-mode "org" "Org mode" t)
(autoload 'org-diary "org" "Org mode diary entries")
(autoload 'org-agenda "org" "Agenda from Org files" t)
(autoload 'org-store-link "org" "Store org link" t)
(autoload 'orgtbl-mode "org" "Orgtbl minor mode" t)
(autoload 'turn-on-orgtbl "org" "Orgtbl minor mode")
(add-to-list 'auto-mode-alist '("\\\\.org\$" . org-mode))
(define-key global-map "\\C-cl" 'org-store-link)$^2$
(define-key global-map "\\C-ca" 'org-agenda)$^2$
\endexample
%
\metax{For the many customization options try}{M-x org-customize}
\metax{To read the on-line documentation try}{M-x org-info}

\section{Visibility Cycling}

\key{rotate current subtree between states}{TAB}
\key{rotate entire buffer between states}{S-TAB}
\key{show the whole file}{C-c C-a}
%\key{show branches}{C-c C-k}

\section{Motion}

\key{next heading}{C-c C-n}
\key{previous heading}{C-c C-p}
\key{next heading, same level}{C-c C-f}
\key{previous heading, same level}{C-c C-b}
\key{backward to higher level heading}{C-c C-u}
\key{jump to another place in document}{C-c C-j}

\section{Structure Editing}

\key{insert new heading at same level as current}{M-RET}
\key{insert new TODO entry}{M-S-RET}

\key{promote current heading up one level}{M-LEFT}
\key{demote current heading down one level}{M-RIGHT}
\key{promote current subtree up one level}{M-S-LEFT}
\key{demote current subtree down one level}{M-S-RIGHT}

\key{move subtree up}{M-S-UP}
\key{move subtree down}{M-S-DOWN}
\key{kill subtree}{C-c C-x C-w}
\key{copy subtree}{C-c C-x M-w}
\key{yank subtree}{C-c C-x C-y}

\key{archive subtree}{C-c \$}
To set archive location for current file, add a line like$^3$:
\vskip -1mm
\beginexample%
\#+ARCHIVE: archfile::* Archived Tasks
\endexample

\section{Filtering and Sparse Trees}

\key{show sparse tree for all matches of a regexp}{C-c /}
\key{view TODO's in sparse tree}{C-c C-v}
\key{create sparse tree with all deadlines due}{C-c C-w}
\key{time sorted view of current org file}{C-c C-r}
\key{agenda for the week}{C-c a$1$}
\key{agenda for date at cursor}{C-c C-o}

\section{TODO Items}

\key{rotate the state of the current item}{C-c C-t}
\key{view TODO items in a sparse tree}{C-c C-v}

\key{set the priority of the current item}{C-c , [ABC]}
\key{remove priority cookie from current item}{C-c , SPC}
\key{raise priority of current item}{S-UP$^4$}
\key{lower priority of current item}{S-DOWN$^4$}

\vskip 1mm
per-file TODO workflow states: add line(s) like$^3$:
\vskip -1mm
\beginexample%
\#+SEQ_TODO: TODO PROCRASTINATE BLUFF DONE
\endexample
\vskip -1mm
per-file TODO keywords: add line(s) like$^3$:
\vskip -1mm
\beginexample%
\#+TYP_TODO: Phil home work DONE
\endexample

\section{Timestamps}

\key{prompt for date and insert timestamp}{C-c .}
\key{like \kbd{C-c} . but insert date and time format}{C-u C-c .}
\key{Like \kbd{C-c .} but make stamp inactive}{C-c !} % FIXME
\key{insert DEADLINE timestamp}{C-c C-d}
\key{insert SCHEDULED timestamp}{C-c C-s}
\key{create sparse tree with all deadlines due}{C-c C-w}
\key{the time between 2 dates in a time range}{C-c C-y}
\key{change timestamp at cursor by $-1$ day}{S-LEFT$^4$}
\key{change timestamp at cursor by $+1$ day}{S-RIGHT$^4$}
\key{change year/month/day at cursor by $-1$}{S-DOWN$^4$}
\key{change year/month/day at cursor by $+1$}{S-UP$^4$}
\key{access the calendar for the current date}{C-c >}
\key{insert timestamp matching date in calendar}{C-c <}
\key{access agenda for current date}{C-c C-o}
\key{While prompted for a date:}{}
\key{... select date in calendar}{mouse-1/RET}
\key{... scroll calendar back/forward one month}{< / >}
\key{... forward/backward one day}{S-LEFT/RIGHT}
\key{... forward/backward one week}{S-UP/DOWN}
\key{... forward/backward one month}{M-S-LEFT/RIGT}

\section{Links}

\key{globally store link to the current location}{C-c l$^2$}
\key{insert a link (TAB completes stored links)}{C-c C-l}
\key{insert file link with file name completion}{C-u C-c C-l}

\key{open link at point}{C-c C-o}
\key{open file links in emacs}{C-u C-c C-o}
\key{open link at point}{mouse-2}
\key{open file links in emacs}{mouse-3}

{\bf Link types}

\key{\kbd{<http://www.astro.uva.nl/~dominik>}}{\rm on the web}
\key{\kbd{<mailto:adent@galaxy.net>}}{\rm EMail address}
\key{\kbd{<news:comp.emacs>}}{\rm Usenet group}
\key{\kbd{<file:/home/dominik/img/mars.jpg>}}{\rm file, absolute}
\key{\kbd{<file:papers/last.pdf>}}{\rm file, relative}
\key{\kbd{<file:~/code/main.c:255>}}{\rm file with line nr.}
\key{\kbd{<bbdb:Richard Stallman>}}{\rm BBDB person}
\key{\kbd{<shell:ls *.org>}}{\rm shell command}
\key{\kbd{<gnus:group>}}{\rm GNUS group}
\key{\kbd{<gnus:group\#id>}}{\rm GNUS message}
\key{\kbd{<vm:folder>}}{\rm VM folder}
\key{\kbd{<vm:folder\#id>}}{\rm VM message}
\key{\kbd{<vm://myself@some.where.org/folder\#id>}}{\rm VM remote}
Wanderlust \kbd{<wl:...>} and RMAIL \kbd{<rmail:...>} like VM
%\key{\kbd{<wl:folder>}}{\rm Wanderlust f.}
%\key{\kbd{<wl:folder\#id>}}{\rm Wanderlust m.}
%\key{\kbd{<rmail:folder>}}{\rm RMAIL folder}
%\key{\kbd{<rmail:folder\#id>}}{\rm RMAIL msg}

\section{Tables}

%Org-mode has its own built-in intuitive table editor with unique
%capabilities.

{\bf Creating a table}

\metax{insert a new Org-mode table}{M-x org-table-create}
\metax{... or just start typing, e.g.}{|Name|Phone|Age RET |- TAB}
\key{convert region to table}{C-c C-c}
\key{... separator at least 3 spaces}{C-3 C-c C-c}
%\key{... prompt for separator regexp}{C-u C-c C-c}

{\bf Commands available inside tables}

The following commands work when the cursor is {\it inside a table}.
Outside of tables, the same keys may have other functionality.

{\bf Re-aligning and field motion}

\key{re-align the table without moving the cursor}{C-c C-c}
\key{re-align the table, move to next field}{TAB}
\key{move to previous field}{S-TAB}
\key{re-align the table, move to next row}{RET}

{\bf Row and column editing}

\key{move the current column left}{M-LEFT}
\key{move the current column right}{M-RIGHT}
\key{kill the current column}{M-S-LEFT}
\key{insert new column to left of cursor position}{M-S-RIGHT}

\key{move the current row up}{M-UP}
\key{move the current row down}{M-DOWN}
\key{kill the current row or horizontal line}{M-S-UP}
\key{insert new row above the current row}{M-S-DOWN}

\key{insert horizontal line below the current row}{C-c -}
\key{insert horizontal line above the current row}{C-u C-c -}

{\bf Regions}

\key{cut rectangular region}{C-c C-x C-w}
\key{copy rectangular region}{C-c C-x M-w}
\key{paste rectangular region}{C-c C-x C-y}
\key{fill paragraph across selected cells}{C-c C-q}

{\bf Calculations}

Except for the summation commands, these need the Emacs calc package
installed.

\key{set and eval column formula}{C-c =}
\key{set and eval named-field formula}{C-u C-c =}
\key{edit formulas in separate buffer}{C-c '}
\key{re-apply all stored equations to current line}{C-c *}
\key{re-apply all stored equations to entire table}{C-u C-c *}

\kbd{TAB}, \kbd{RET} and \kbd{C-c C-c} trigger automatic recalculation
in lines starting with: {\tt | \# |}.

\key{rotate calculation mark through \# * ! \^ \_ \$}{C-\#}

\key{display column number cursor is in}{C-c ?}
\key{sum numbers in current column/rectangle}{C-c +}
\key{copy down with increment}{S-RET$^4$}

A formula can also be typed directly into into a field and will
executed by \kbd{TAB}, \kbd{RET} and \kbd{C-c C-c}.  A leading \kbd{=}
introduces a column formula, \kbd{:=} a named-field formula.

\key{Example: Add Col1 and Col2}{=\$1+\$2}
\key{... with printf format specification}{=\$1+\$2;\%.2f}
\key{... with constants from constants.el}{=\$1/\$c/\$cm}
\key{sum from 3rd hline above to here}{:=vsum(\&III)}
\key{apply current column formula}{=}

{\bf Miscellaneous}

\key{toggle visibility of vertical lines}{C-c |}
\metax{export as tab-separated file}{M-x org-table-export}
\metax{import tab-separated file}{M-x org-table-import}

{\bf Tables created with the \kbd{table.el} package}

\key{insert a new \kbd{table.el} table}{C-c ~}
\key{recognize existing table.el table}{C-c C-c}
\key{convert table (Org-mode $\leftrightarrow$ table.el)}{C-c ~}

\newcolumn
\title{Org-Mode Reference Card (2/2)}

\centerline{(for version \orgversionnumber)}

\section{Timeline and Agenda}

\key{show timeline of current org file}{C-c C-r}
\key{... include past dates}{C-u C-c C-r}

\key{add current file to your agenda}{C-c [}
\key{remove current file from your agenda}{C-c ]}
\key{compile agenda for the current week}{C-c a$^2$}
\key{agenda for date at cursor}{C-c C-o}
\vskip 1mm
To set category for current file, add line$^3$:
\vskip -1mm
\beginexample%
\#+CATEGORY: MyCateg
\endexample

{\bf Commands available in an agenda buffer}

The agenda buffer is electric: single key presses execute commands.

{\bf View org file}

\key{show original location of item}{SPC}
\key{... also available with}{mouse-3}
\key{show and recenter window}{l}
\key{goto original location in other window}{TAB}
\key{... also available with}{mouse-2}
\key{goto original location, delete other windows}{RET}
\key{toggle follow-mode}{f}

{\bf Change display}

\key{delete other windows}{o}
\key{switch to weekly view}{w}
\key{switch to daily view}{d}
\key{toggle inclusion of diary entries}{D}
\key{toggle time grid for daily schedule}{g}
\key{refresh agenda buffer with any changes}{r}
\key{display the following \kbd{org-agenda-ndays}}{RIGHT}
\key{display the previous \kbd{org-agenda-ndays}}{LEFT}
\key{goto today}{.}

{\bf Remote editing}

\key{digit argument}{0-9}

\key{change state of current TODO item}{t}
\key{set priority of current item}{p}
\key{raise priority of current item}{S-UP$^4$}
\key{lower priority of current item}{S-DOWN$^4$}
\key{display weighted priority of current item}{P}

\key{change timestamp to one day earlier}{S-LEFT$^4$}
\key{change timestamp to one day later}{S-RIGHT$^4$}
\key{change timestamp to today}{>}

\key{insert new entry into diary}{i}

{\bf Calendar commands}

\key{find agenda cursor date in calendar}{c}
\key{compute agenda for calendar cursor date}{c}
\key{show phases of the moon}{M}
\key{show sunrise/sunset times}{S}
\key{show holidays}{H}
\key{convert date to other calendars}{C}

{\bf Quit and Exit}

\key{quit agenda, remove agenda buffer}{q}
\key{exit agenda, remove all agenda buffers}{x}

\section{Exporting}

Exporting creates files with extensions {\it .txt\/} and {\it .html\/}
in the current directory.

\key{export as ASCII file}{C-c C-x a}
\key{export visible text only (e.g. for printing)}{C-c C-x v}
\key{export as HTML file}{C-c C-x h}
\key{export as HTML and open in browser}{C-c C-x b}
\key{prefix arg sets nb. of headline levels, e.g.}{C-3 C-c C-x h}

\key{insert template of export options}{C-c C-x t}

\key{toggle fixed width for entry or region}{C-c :}

{\bf HTML formatting}

\key{make words {\bf bold}}{*bold*}
\key{make words {\it italic}}{/italic/}
\key{make words \underbar{underlined}}{_underlined_}
\key{sub- and superscripts}{x\^{}3, J_dust}
\key{\TeX{}-like macros}{\\alpha, \\to}
\key{typeset lines in fixed width font}{start with :}
\key{tables are exported as HTML tables}{start with |}
\key{links become HTML links}{http:... etc}
\key{include html tags}{@<b>...@</b>}

{\bf Export options}

Include additional information for export by putting these anywhere in the
org file.  Use {\tt M-TAB} completion to make sure to get the right
keywords. {\tt M-TAB} again just after keyword is complete inserts examples.

\key{the title to be shown}{\#+TITLE:}
\key{the author}{\#+AUTHOR:}
\key{authors email address}{\#+EMAIL:}
\key{language code for html}{\#+LANGUAGE:}
\key{free text description of file}{\#+TEXT:}
\key{... which can carry over multiple lines}{\#+TEXT:}
\key{settings for the export process - see below}{\#+OPTIONS:}

{\bf Settings for the OPTIONS line}

\key{set number of headline levels for export}{H:2}
\key{turn on/off section numbers}{num:t}
\key{turn on/off table of contents}{toc:t}
\key{turn on/off linebreak preservation}{\\n:nil}
\key{turn on/off quoted html tags}{@:t}
\key{turn on/off fixed width sections}{::t}
\key{turn on/off tables}{|:t}
\key{turn on/off \TeX\ syntax for sub/super-scripts}{\^{}:t}
\key{turn on/off emphasised text}{*:nil}
\key{turn on/off \TeX\ macros}{TeX:t}

{\bf Comments: Text not being exported}

Text before the first headline is not considered part of the document
and is therefore never exported.
Lines starting with \kbd{\#} are comments and are not exported.
Subtrees whose header starts with COMMENT are never exported.

\key{toggle COMMENT keyword on entry}{C-c ;}

\section{Completion}

In-buffer completion completes TODO keywords at headline start, TeX
macros after ``{\tt \\}'', option keywords after ``{\tt \#-}'', and
dictionary words elsewhere.

\key{Complete word at point}{M-TAB}

\newcolumn

\section{Calendar and Diary Integration}

To include entries from the Emacs diary in your Org-mode agenda:
\beginexample%
(setq org-agenda-include-diary t)
\endexample

To include your Org-mode agenda in your normal diary, make sure you're
using the fancy diary display
%
%\beginexample%
%(add-hook 'diary-display-hook 'fancy-diary-display)
%\endexample
%
and in your \kbd{~/diary} file add

\beginexample%
\&\%\%(org-diary)
\endexample

to include all the files listed in \kbd{org-agenda-files}.  For more
selective file inclusion use a line for each file:

\beginexample%
\&\%\%(org-diary) ~/path/to/some/org-file.org
\endexample

\section{Remember-mode Integration}

See the manual for how to make remember.el use Org-mode links and
files.  The note-finishing command \kbd{C-c C-c} will first prompt for
an org file. In the file, find a location with:

\key{rotate subtree visibility}{TAB}
\key{next heading}{DOWN}
\key{previous heading}{UP}

Insert the note with one of the following: 

\key{as sublevel of heading at cursor}{RET}
\key{right here (cursor not on heading)}{RET}
\key{before current heading}{LEFT}
\key{after current heading}{RIGHT}
\key{shortcut to end of buffer (cursor at buf-start)}{RET}
\key{Abort}{q}

\section{CUA and pc-select compatibility}

Configure the variable {\tt org-CUA-compatibility} to make Org-mode
avoid the \kbd{S-<cursor>} bindings used by these modes.  When set,
Org-mode will change the following keybindings (also in the agenda
buffer, but not during date selection). See note mark four$^4$
throughout the reference card.
%\vskip-mm
\beginexample
S-UP    $\to$ M-p             S-DOWN  $\to$ M-n
S-LEFT  $\to$ M--             S-RIGHT $\to$ M-+
S-RET   $\to$ C-S-RET
\endexample

\section{Notes}
$^1$ The six autoload forms are only needed if Org-mode is not part of
Emacs, or an XEmacs package.

$^2$ This is only a suggestion for a binding of this command.  Choose
you own key as shown under INSTALLATION.

$^3$ After changing a \kbd{\#+KEYWORD} line, press \kbd{C-c C-c} with
the cursor still in a line to make Org-mode notice the change.

$^4$ Keybinding affected by {\tt org-CUA-compatibility}.

\copyrightnotice

\bye


% arch-tag: 139f6750-5cfc-49ca-92b5-237fe5795290

