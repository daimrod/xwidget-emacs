% Reference Card for Org Mode
\def\orgversionnumber{7.3}
\def\versionyear{2010}          % latest update
\def\year{2010}                 % latest copyright year

%**start of header
\newcount\columnsperpage
\newcount\letterpaper

% This file can be printed with 1, 2, or 3 columns per page (see below).
% Specify how many you want here.
\columnsperpage=3

% PDF output layout.  0 for A4, 1 for letter (US), a `l' is added for
% a landscape layout.
\input pdflayout.sty
\pdflayout=(0l)

% Nothing else needs to be changed below this line.
% Copyright (C) 1987, 1993, 1996, 1997, 2001, 2002, 2003, 2004, 2005,
%   2006, 2007, 2008, 2009  Free Software Foundation, Inc.

% This file is part of GNU Emacs.

% GNU Emacs is free software: you can redistribute it and/or modify
% it under the terms of the GNU General Public License as published by
% the Free Software Foundation, either version 3 of the License, or
% (at your option) any later version.

% GNU Emacs is distributed in the hope that it will be useful,
% but WITHOUT ANY WARRANTY; without even the implied warranty of
% MERCHANTABILITY or FITNESS FOR A PARTICULAR PURPOSE.  See the
% GNU General Public License for more details.

% You should have received a copy of the GNU General Public License
% along with GNU Emacs.  If not, see <http://www.gnu.org/licenses/>.

% This file is intended to be processed by plain TeX (TeX82).
%
% The final reference card has six columns, three on each side.
% This file can be used to produce it in any of three ways:
% 1 column per page
%    produces six separate pages, each of which needs to be reduced to 80%.
%    This gives the best resolution.
% 2 columns per page
%    produces three already-reduced pages.
%    You will still need to cut and paste.
% 3 columns per page
%    produces two pages which must be printed sideways to make a
%    ready-to-use 8.5 x 11 inch reference card.
%    For this you need a dvi device driver that can print sideways.
% Which mode to use is controlled by setting \columnsperpage above.
%
% To compile and print this document:
% tex refcard.tex
% dvips -t landscape refcard.dvi
%
% Author:
%  Stephen Gildea
%  Internet: gildea@stop.mail-abuse.org
%
% Thanks to Paul Rubin, Bob Chassell, Len Tower, and Richard Mlynarik
% for their many good ideas.

\def\shortcopyrightnotice{\vskip 1ex plus 2 fill
  \centerline{\small \copyright\ \year\ Free Software Foundation, Inc.
  Permissions on back.  v\orgversionnumber}}

\def\copyrightnotice{
\vskip 1ex plus 100 fill\begingroup\small
\centerline{Copyright \copyright\ \year\ Free Software Foundation, Inc.}
\centerline{v\orgversionnumber{} for Org-Mode \orgversionnumber{}, \versionyear}
\centerline{Author: Philip Rooke}
\centerline{based on refcard design and format by Stephen Gildea}

Permission is granted to make and distribute copies of
this card provided the copyright notice and this permission notice
are preserved on all copies.

\endgroup}

% make \bye not \outer so that the \def\bye in the \else clause below
% can be scanned without complaint.
\def\bye{\par\vfill\supereject\end}

\newdimen\intercolumnskip	%horizontal space between columns
\newbox\columna			%boxes to hold columns already built
\newbox\columnb

\def\ncolumns{\the\columnsperpage}

\message{[\ncolumns\space
  column\if 1\ncolumns\else s\fi\space per page]}

\def\scaledmag#1{ scaled \magstep #1}

% This multi-way format was designed by Stephen Gildea October 1986.
% Note that the 1-column format is fontfamily-independent.
\if 1\ncolumns			%one-column format uses normal size
  \hsize 4in
  \vsize 10in
  \voffset -.7in
  \font\titlefont=\fontname\tenbf \scaledmag3
  \font\headingfont=\fontname\tenbf \scaledmag2
  \font\smallfont=\fontname\sevenrm
  \font\smallsy=\fontname\sevensy

  \footline{\hss\folio}
  \def\makefootline{\baselineskip10pt\hsize6.5in\line{\the\footline}}
\else				%2 or 3 columns uses prereduced size
  \if 1\the\letterpaper
     \hsize 3.2in
     \vsize 7.95in
     \hoffset -.75in
     \voffset -.745in
  \else
     \hsize 3.2in
     \vsize 7.65in
     \hoffset -.25in
     \voffset -.745in
  \fi
  \font\titlefont=cmbx10 \scaledmag2
  \font\headingfont=cmbx10 \scaledmag1
  \font\smallfont=cmr6
  \font\smallsy=cmsy6
  \font\eightrm=cmr8
  \font\eightbf=cmbx8
  \font\eightit=cmti8
  \font\eighttt=cmtt8
  \font\eightmi=cmmi8
  \font\eightsy=cmsy8
  \textfont0=\eightrm
  \textfont1=\eightmi
  \textfont2=\eightsy
  \def\rm{\eightrm}
  \def\bf{\eightbf}
  \def\it{\eightit}
  \def\tt{\eighttt}
  \if 1\the\letterpaper
     \normalbaselineskip=.8\normalbaselineskip
  \else
     \normalbaselineskip=.7\normalbaselineskip
  \fi
  \normallineskip=.8\normallineskip
  \normallineskiplimit=.8\normallineskiplimit
  \normalbaselines\rm		%make definitions take effect

  \if 2\ncolumns
    \let\maxcolumn=b
    \footline{\hss\rm\folio\hss}
    \def\makefootline{\vskip 2in \hsize=6.86in\line{\the\footline}}
  \else \if 3\ncolumns
    \let\maxcolumn=c
    \nopagenumbers
  \else
    \errhelp{You must set \columnsperpage equal to 1, 2, or 3.}
    \errmessage{Illegal number of columns per page}
  \fi\fi

  \intercolumnskip=.46in
  \def\abc{a}
  \output={%			%see The TeXbook page 257
      % This next line is useful when designing the layout.
      %\immediate\write16{Column \folio\abc\space starts with \firstmark}
      \if \maxcolumn\abc \multicolumnformat \global\def\abc{a}
      \else\if a\abc
	\global\setbox\columna\columnbox \global\def\abc{b}
        %% in case we never use \columnb (two-column mode)
        \global\setbox\columnb\hbox to -\intercolumnskip{}
      \else
	\global\setbox\columnb\columnbox \global\def\abc{c}\fi\fi}
  \def\multicolumnformat{\shipout\vbox{\makeheadline
      \hbox{\box\columna\hskip\intercolumnskip
        \box\columnb\hskip\intercolumnskip\columnbox}
      \makefootline}\advancepageno}
  \def\columnbox{\leftline{\pagebody}}

  \def\bye{\par\vfill\supereject
    \if a\abc \else\null\vfill\eject\fi
    \if a\abc \else\null\vfill\eject\fi
    \end}
\fi

% we won't be using math mode much, so redefine some of the characters
% we might want to talk about
%\catcode`\^=12
\catcode`\_=12

% we also need the tilde, for file names.
\catcode`\~=12

\chardef\\=`\\
\chardef\{=`\{
\chardef\}=`\}

\hyphenation{mini-buf-fer}

\parindent 0pt
\parskip 1ex plus .5ex minus .5ex

\def\small{\smallfont\textfont2=\smallsy\baselineskip=.8\baselineskip}

% newcolumn - force a new column.  Use sparingly, probably only for
% the first column of a page, which should have a title anyway.
\outer\def\newcolumn{\vfill\eject}

% title - page title.  Argument is title text.
\outer\def\title#1{{\titlefont\centerline{#1}}\vskip 1ex plus .5ex}

% section - new major section.  Argument is section name.
\outer\def\section#1{\par\filbreak
  \vskip 3ex plus 2ex minus 2ex {\headingfont #1}\mark{#1}%
  \vskip 2ex plus 1ex minus 1.5ex}

\newdimen\keyindent

% beginindentedkeys...endindentedkeys - key definitions will be
% indented, but running text, typically used as headings to group
% definitions, will not.
\def\beginindentedkeys{\keyindent=1em}
\def\endindentedkeys{\keyindent=0em}
\endindentedkeys

% paralign - begin paragraph containing an alignment.
% If an \halign is entered while in vertical mode, a parskip is never
% inserted.  Using \paralign instead of \halign solves this problem.
\def\paralign{\vskip\parskip\halign}

% \<...> - surrounds a variable name in a code example
\def\<#1>{{\it #1\/}}

% kbd - argument is characters typed literally.  Like the Texinfo command.
\def\kbd#1{{\tt#1}\null}	%\null so not an abbrev even if period follows

% beginexample...endexample - surrounds literal text, such a code example.
% typeset in a typewriter font with line breaks preserved
\def\beginexample{\par\leavevmode\begingroup
  \obeylines\obeyspaces\parskip0pt\tt}
{\obeyspaces\global\let =\ }
\def\endexample{\endgroup}

% key - definition of a key.
% \key{description of key}{key-name}
% prints the description left-justified, and the key-name in a \kbd
% form near the right margin.
\def\key#1#2{\leavevmode\hbox to \hsize{\vtop
  {\hsize=.75\hsize\rightskip=1em
  \hskip\keyindent\relax#1}\kbd{#2}\hfil}}

\newbox\metaxbox
\setbox\metaxbox\hbox{\kbd{M-x }}
\newdimen\metaxwidth
\metaxwidth=\wd\metaxbox

% metax - definition of a M-x command.
% \metax{description of command}{M-x command-name}
% Tries to justify the beginning of the command name at the same place
% as \key starts the key name.  (The "M-x " sticks out to the left.)
\def\metax#1#2{\leavevmode\hbox to \hsize{\hbox to .75\hsize
  {\hskip\keyindent\relax#1\hfil}%
  \hskip -\metaxwidth minus 1fil
  \kbd{#2}\hfil}}

% threecol - like "key" but with two key names.
% for example, one for doing the action backward, and one for forward.
\def\threecol#1#2#3{\hskip\keyindent\relax#1\hfil&\kbd{#2}\hfil\quad
  &\kbd{#3}\hfil\quad\cr}

%**end of header


\title{Org-Mode Reference Card (1/2)}

\centerline{(for version \orgversionnumber)}

\section{Getting Started}
%
%\vskip -2mm
%\beginexample%
%(add-to-list 'auto-mode-alist '("\\\\.org\$" . org-mode))
%(define-key global-map "\\C-cl" 'org-store-link)$^1$
%(define-key global-map "\\C-ca" 'org-agenda)$^1$
%\endexample
%
\metax{To read the on-line documentation try}{M-x org-info}

\section{Visibility Cycling}

\key{rotate current subtree between states}{TAB}
\key{rotate entire buffer between states}{S-TAB}
\key{restore property-dependent startup visibility}{C-u C-u TAB}
\metax{show the whole file, including drawers}{C-u C-u C-u TAB}
\key{reveal context around point}{C-c C-r}

\section{Motion}

\key{next/previous heading}{C-c C-n/p}
\key{next/previous heading, same level}{C-c C-f/b}
\key{backward to higher level heading}{C-c C-u}
\key{jump to another place in document}{C-c C-j}
\key{previous/next plain list item}{S-UP/DOWN$^3$}

\section{Structure Editing}

\key{insert new heading/item at current level}{M-RET}
\key{insert new heading after subtree}{C-RET}
\key{insert new TODO entry/checkbox item}{M-S-RET}
\key{insert TODO entry/ckbx after subtree}{C-S-RET}
\key{turn (head)line into item, cycle item type}{C-c -}
\key{turn item/line into headline}{C-c *}
\key{promote/demote heading}{M-LEFT/RIGHT}
\metax{promote/demote current subtree}{M-S-LEFT/RIGHT}

\metax{move subtree/list item up/down}{M-S-UP/DOWN}
\metax{sort subtree/region/plain-list}{C-c \^{}}
\metax{clone a subtree}{C-c C-x c}
\metax{refile subtree}{C-c C-w}
\metax{kill/copy subtree}{C-c C-x C-w/M-w}
\metax{yank subtree}{C-c C-x C-y or C-y}
\metax{narrow buffer to subtree / widen}{C-x n s/w}

\section{Archiving}

\key{archive subtree using the default command}{C-c C-x C-a}
\key{move subtree to archive file}{C-c C-x C-s}
\key{toggle ARCHIVE tag / to ARCHIVE sibling}{C-c C-x a/A}
\key{force cycling of an ARCHIVEd tree}{C-TAB}

\section{Filtering and Sparse Trees}

\key{construct a sparse tree by various criteria}{C-c /}
\key{view TODO's in sparse tree}{C-c / t/T}
\key{global TODO list in agenda mode}{C-c a t$^1$}
\key{time sorted view of current org file}{C-c a L}

\section{Tables}

{\bf Creating a table}

%\metax{insert a new Org-mode table}{M-x org-table-create}
\metax{just start typing, e.g.}{|Name|Phone|Age RET |- TAB}
\key{convert region to table}{C-c |}
\key{... separator at least 3 spaces}{C-3 C-c |}

{\bf Commands available inside tables}

The following commands work when the cursor is {\it inside a table}.
Outside of tables, the same keys may have other functionality.

{\bf Re-aligning and field motion}

\key{re-align the table without moving the cursor}{C-c C-c}
\key{re-align the table, move to next field}{TAB}
\key{move to previous field}{S-TAB}
\key{re-align the table, move to next row}{RET}
\key{move to beginning/end of field}{M-a/e}

{\bf Row and column editing}

\key{move the current column left}{M-LEFT/RIGHT}
\key{kill the current column}{M-S-LEFT}
\key{insert new column to left of cursor position}{M-S-RIGHT}

\key{move the current row up/down}{M-UP/DOWN}
\key{kill the current row or horizontal line}{M-S-UP}
\key{insert new row above the current row}{M-S-DOWN}
\key{insert hline below (\kbd{C-u} : above) current row}{C-c -}
\key{insert hline and move to line below it}{C-c RET}
\key{sort lines in region}{C-c \^{}}

{\bf Regions}

\metax{cut/copy/paste rectangular region}{C-c C-x C-w/M-w/C-y}
%\key{copy rectangular region}{C-c C-x M-w}
%\key{paste rectangular region}{C-c C-x C-y}
\key{fill paragraph across selected cells}{C-c C-q}

{\bf Miscellaneous}

\key{to limit column width to \kbd{N} characters, use}{...| <N> |...}
\key{edit the current field in a separate window}{C-c `}
\key{make current field fully visible}{C-u TAB}
\metax{export as tab-separated file}{M-x org-table-export}
\metax{import tab-separated file}{M-x org-table-import}
\key{sum numbers in current column/rectangle}{C-c +}

{\bf Tables created with the \kbd{table.el} package}

\key{insert a new \kbd{table.el} table}{C-c ~}
\key{recognize existing table.el table}{C-c C-c}
\key{convert table (Org-mode $\leftrightarrow$ table.el)}{C-c ~}

{\bf Spreadsheet}

Formulas typed in field are executed by \kbd{TAB},
\kbd{RET} and \kbd{C-c C-c}.  \kbd{=} introduces a column
formula, \kbd{:=} a field formula.

\key{Example: Add Col1 and Col2}{|=\$1+\$2      |}
\key{... with printf format specification}{|=\$1+\$2;\%.2f|}
\key{... with constants from constants.el}{|=\$1/\$c/\$cm |}
\metax{sum from 2nd to 3rd hline}{|:=vsum(@II..@III)|}
\key{apply current column formula}{| = |}

\key{set and eval column formula}{C-c =}
\key{set and eval field formula}{C-u C-c =}
\key{re-apply all stored equations to current line}{C-c *}
\key{re-apply all stored equations to entire table}{C-u C-c *}
\key{iterate table to stability}{C-u C-u C-c *}
\key{rotate calculation mark through \# * ! \^ \_ \$}{C-\#}
\key{show line, column, formula reference}{C-c ?}
\key{toggle grid / debugger}{C-c \}/\{}

\newcolumn
{\it Formula Editor}

\key{edit formulas in separate buffer}{C-c '}
\key{exit and install new formulas}{C-c C-c}
\key{exit, install, and apply new formulas}{C-u C-c C-c}
\key{abort}{C-c C-q}
\key{toggle reference style}{C-c C-r}
\key{pretty-print Lisp formula}{TAB}
\key{complete Lisp symbol}{M-TAB}
\key{shift reference point}{S-cursor}
\key{shift test line for column references}{M-up/down}
\key{scroll the window showing the table}{M-S-up/down}
\key{toggle table coordinate grid}{C-c \}}

\section{Links}

\key{globally store link to the current location}{C-c l$^1$}
\key{insert a link (TAB completes stored links)}{C-c C-l}
\key{insert file link with file name completion}{C-u C-c C-l}
\key{edit (also hidden part of) link at point}{C-c C-l}

\key{open file links in emacs}{C-c C-o}
\key{...force open in emacs/other window}{C-u C-c C-o}
\key{open link at point}{mouse-1/2}
\key{...force open in emacs/other window}{mouse-3}
\key{record a position in mark ring}{C-c \%}
\key{jump back to last followed link(s)}{C-c \&}
\key{find next link}{C-c C-x C-n}
\key{find previous link}{C-c C-x C-p}
\key{edit code snippet of file at point}{C-c '}
\key{toggle inline display of linked images}{C-c C-x C-v}

% {\bf Internal Links}

% \key{\kbd{<<My Target>>}}{\rm target}
% \key{\kbd{<<<My Target>>>}}{\rm radio target$^2$}
% \key{\kbd{[[*this text]]}}{\rm find headline}
% \metax{\kbd{[[this text]]}}{\rm find target or text in buffer}
% \metax{\kbd{[[this text][description]]}}{\rm optional link text}

% {\bf External Links}

% \key{\kbd{file:/home/dominik/img/mars.jpg}}{\rm file, absolute}
% \key{\kbd{file:papers/last.pdf}}{\rm file, relative}
% \key{\kbd{file:projects.org::*that text}}{\rm find headline}
% \key{\kbd{file:projects.org::find me}}{\rm find trgt/string}
% %\key{\kbd{file:projects.org::/regexp/}}{\rm regexp search}
% \key{\kbd{http://www.astro.uva.nl/~dominik}}{\rm on the web}
% \key{\kbd{mailto:adent@galaxy.net}}{\rm Email address}
% \key{\kbd{news:comp.emacs}}{\rm Usenet group}
% \key{\kbd{bbdb:Richard Stallman}}{\rm BBDB person}
% \key{\kbd{gnus:group}}{\rm GNUS group}
% \key{\kbd{gnus:group\#id}}{\rm GNUS message}
% \key{\kbd{vm|wl|mhe|rmail:folder}}{\rm Mail folder}
% \key{\kbd{vm|wl|mhe|rmail:folder\#id}}{\rm Mail message}
% \key{\kbd{info:emacs:Regexps}}{\rm Info file:node}
% \key{\kbd{shell:ls *.org}}{\rm shell command}
% \key{\kbd{elisp:(calendar)}}{\rm elisp form}
% \metax{\kbd{[[external link][description]]}}{\rm optional link text}
% %\key{\kbd{vm://myself@some.where.org/folder\#id}}{\rm VM remote}

\section{Working with Code (Babel)}

\key{execute code block at point}{C-c C-c}
\key{open results of code block at point}{C-c C-o}
\key{view expanded body of code block at point}{C-c C-v v}
\key{go to named code block}{C-c C-v g}
\key{go to named result}{C-c C-v r}
\key{go to the head of the current code block}{C-c C-v u}
\key{go to the next code block}{C-c C-v n}
\key{go to the previous code block}{C-c C-v p}
\key{demarcate a code block}{C-c C-v d}
\key{execute the next key sequence in the code edit buffer}{C-c C-v x}
\key{execute all code blocks in current buffer}{C-c C-v b}
\key{execute all code blocks in current subtree}{C-c C-v s}
\key{tangle code blocks in current file}{C-c C-v t}
\key{tangle code blocks in supplied file}{C-c C-v f}
\key{ingest all code blocks in supplied file into the Library of Babel}{C-c C-v i}
\key{switch to the session of the current code block}{C-c C-v z}
\key{load expanded body of the current code block into a session}{C-c C-v l}
\key{view sha1 hash of the current code block}{C-c C-v a}

% \section{Remember-mode Integration}

% See the manual for how to make remember.el use Org-mode links and
% files.  The note-finishing command \kbd{C-c C-c} will first prompt for
% an org file. In the file, find a location with:

% \key{rotate subtree visibility}{TAB}
% \key{next heading}{DOWN}
% \key{previous heading}{UP}

% Insert the note with one of the following: 

% \key{as sublevel of heading at cursor}{RET}
% \key{right here (cursor not on heading)}{RET}
% \key{before current heading}{LEFT}
% \key{after current heading}{RIGHT}
% \key{shortcut to end of buffer (cursor at buf-start)}{RET}
% \key{Abort}{q}

\section{Completion}

In-buffer completion completes TODO keywords at headline start, TeX
macros after ``{\tt \\}'', option keywords after ``{\tt \#-}'', TAGS
after  ``{\tt :}'', and dictionary words elsewhere.

\key{complete word at point}{M-TAB}


\newcolumn
\title{Org-Mode Reference Card (2/2)}

\centerline{(for version \orgversionnumber)}

\section{TODO Items and Checkboxes}

\key{rotate the state of the current item}{C-c C-t}
\metax{select next/previous state}{S-LEFT/RIGHT}
\metax{select next/previous set}{C-S-LEFT/RIGHT}
\key{toggle ORDERED property}{C-c C-x o}
\key{view TODO items in a sparse tree}{C-c C-v}
\key{view 3rd TODO keyword's sparse tree}{C-3 C-c C-v}

\key{set the priority of the current item}{C-c , [ABC]}
\key{remove priority cookie from current item}{C-c , SPC}
\key{raise/lower priority of current item}{S-UP/DOWN$^3$}
%\key{lower priority of current item}{S-DOWN$^3$}

%\key{\kbd{\#+SEQ_TODO: TODO TRY BLUFF DONE}}{\rm todo workflow}
%\key{\kbd{\#+TYP_TODO: Phil home work DONE}}{\rm todo types}

\key{insert new checkbox item in plain list}{M-S-RET}
\key{toggle checkbox(es) in region/entry/at point}{C-c C-x C-b}
\key{toggle checkbox at point}{C-c C-c}
%\metax{checkbox statistics cookies: insert {\tt [/]} or {\tt [\%]}}{}
\key{update checkbox statistics (\kbd{C-u} : whole file)}{C-c \#}

\section{Tags}

\key{set tags for current heading}{C-c C-q}
\key{realign tags in all headings}{C-u C-c C-q}
\key{create sparse tree with matching tags}{C-c \\}
\key{globally (agenda) match tags at cursor}{C-c C-o}

\section{Properties and Column View}

\key{set property/effort}{C-c C-x p/e}
\key{special commands in property lines}{C-c C-c}
\key{next/previous allowed value}{S-left/right}
\key{turn on column view}{C-c C-x C-c}
\key{capture columns view in dynamic block}{C-c C-x i}

\key{quit column view}{q}
\key{show full value}{v}
\key{edit value}{e}
\metax{next/previous allowed value}{n/p or S-left/right}
\key{edit allowed values list}{a}
\key{make column wider/narrower}{> / <}
\key{move column left/right}{M-left/right}
\key{add new column}{M-S-right}
\key{Delete current column}{M-S-left}


\section{Timestamps}

\key{prompt for date and insert timestamp}{C-c .}
\key{like \kbd{C-c} . but insert date and time format}{C-u C-c .}
\key{like \kbd{C-c .} but make stamp inactive}{C-c !} % FIXME
\key{insert DEADLINE timestamp}{C-c C-d}
\key{insert SCHEDULED timestamp}{C-c C-s}
\key{create sparse tree with all deadlines due}{C-c / d}
\key{the time between 2 dates in a time range}{C-c C-y}
\key{change timestamp at cursor by $\pm 1$ day}{S-RIGHT/LEFT$^3$}
\key{change year/month/day at cursor by $\pm 1$}{S-UP/DOWN$^3$}
\key{access the calendar for the current date}{C-c >}
\key{insert timestamp matching date in calendar}{C-c <}
\key{access agenda for current date}{C-c C-o}
\key{select date while prompted}{mouse-1/RET}
%\key{... select date in calendar}{mouse-1/RET}
%\key{... scroll calendar back/forward one month}{< / >}
%\key{... forward/backward one day}{S-LEFT/RIGHT}
%\key{... forward/backward one week}{S-UP/DOWN}
%\key{... forward/backward one month}{M-S-LEFT/RIGHT}
\key{toggle custom format display for dates/times}{C-c C-x C-t}

\newcolumn

{\bf Clocking time}

\key{start clock on current item}{C-c C-x C-i}
\key{stop/cancel clock on current item}{C-c C-x C-o/x}
\key{display total subtree times}{C-c C-x C-d}
\key{remove displayed times}{C-c C-c}
\key{insert/update table with clock report}{C-c C-x C-r}

\section{Agenda Views}

\key{add/move current file to front of agenda}{C-c [}
\key{remove current file from your agenda}{C-c ]}
\key{cycle through agenda file list}{C-'}
\key{set/remove restriction lock}{C-c C-x </>}

\key{compile agenda for the current week}{C-c a a$^1$}
\key{compile global TODO list}{C-c a t$^1$}
\key{compile TODO list for specific keyword}{C-c a T$^1$}
\key{match tags, TODO kwds, properties}{C-c a m$^1$}
\key{match only in TODO entries}{C-c a M$^1$}
\key{find stuck projects}{C-c a \#$^1$}
\key{show timeline of current org file}{C-c a L$^1$}
\key{configure custom commands}{C-c a C$^1$}
%\key{configure stuck projects}{C-c a !$^1$}
\key{agenda for date at cursor}{C-c C-o}

{\bf Commands available in an agenda buffer}

{\bf View Org file}

\key{show original location of item}{SPC/mouse-3}
%\key{... also available with}{mouse-3}
\key{show and recenter window}{L}
\key{goto original location in other window}{TAB/mouse-2}
%\key{... also available with}{mouse-2}
\key{goto original location, delete other windows}{RET}
\key{show subtree in indirect buffer, ded.\ frame}{C-c C-x b}
\key{toggle follow-mode}{F}

{\bf Change display}

\key{delete other windows}{o}
\key{view mode dispatcher}{v}
\key{switch to day/week/month/year view}{d w vm vy}
\key{toggle diary entries / time grid / habits}{D / G / K}
\key{toggle entry text / clock report}{E / R}
\key{toggle display of logbook entries}{l / v l/L}
\key{toggle inclusion of archived trees/files}{v a/A}
\key{refresh agenda buffer with any changes}{r / g}
\key{filter with respect to a tag}{/}
\key{save all org-mode buffers}{s}
\key{display next/previous day,week,...}{f / b}
\key{goto today / some date (prompt)}{. / j}

{\bf Remote editing}

\key{digit argument}{0-9}
\key{change state of current TODO item}{t}
\key{kill item and source}{C-k}
\key{archive default}{\$ / a}
\key{refile the subtree}{C-c C-w}
\key{set/show tags of current headline}{: / T}
\key{set effort property (prefix=nth)}{e}
\key{set / compute priority of current item}{, / P}
\key{raise/lower priority of current item}{S-UP/DOWN$^3$}
\key{run an attachment command}{C-c C-a}
\key{schedule/set deadline for this item}{C-c C-s/d}
\key{change timestamp to one day earlier/later}{S-LEFT/RIGHT$^3$}
\key{change timestamp to today}{>}
\key{insert new entry into diary}{i}
\newcolumn
\key{start/stop/cancel the clock on current item}{I / O / X}
\key{jump to running clock entry}{J}
\key{mark / unmark / execute bulk action}{m / u / B}

{\bf Misc}

\key{follow one or offer all links in current entry}{C-c C-o}

{\bf Calendar commands}

\key{find agenda cursor date in calendar}{c}
\key{compute agenda for calendar cursor date}{c}
\key{show phases of the moon}{M}
\key{show sunrise/sunset times}{S}
\key{show holidays}{H}
\key{convert date to other calendars}{C}

{\bf Quit and Exit}

\key{quit agenda, remove agenda buffer}{q}
\key{exit agenda, remove all agenda buffers}{x}

\section{LaTeX and cdlatex-mode}

\key{preview LaTeX fragment}{C-c C-x C-l}
\key{expand abbreviation (cdlatex-mode)}{TAB}
\key{insert/modify math symbol (cdlatex-mode)}{` / '}
\key{insert citation using RefTeX}{C-c C-x [}

\section{Exporting and Publishing}

Exporting creates files with extensions {\it .txt\/} and {\it .html\/}
in the current directory.  Publishing puts the resulting file into
some other place.

\key{export/publish dispatcher}{C-c C-e}

\key{export visible part only}{C-c C-e v}
\key{insert template of export options}{C-c C-e t}
\key{toggle fixed width for entry or region}{C-c :}
\key{toggle pretty display of scripts, entities}{C-c C-x {\tt\char`\\}}

%{\bf HTML formatting}

%\key{make words {\bf bold}}{*bold*}
%\key{make words {\it italic}}{/italic/}
%\key{make words \underbar{underlined}}{_underlined_}
%\key{sub- and superscripts}{x\^{}3, J_dust}
%\key{\TeX{}-like macros}{\\alpha, \\to}
%\key{typeset lines in fixed width font}{start with :}
%\key{tables are exported as HTML tables}{start with |}
%\key{links become HTML links}{http:... etc}
%\key{include html tags}{@<b>...@</b>}

%{\bf Export options}
%
%Include additional information for export by putting these anywhere in the
%org file.  Use {\tt M-TAB} completion to make sure to get the right
%keywords. {\tt M-TAB} again just after keyword is complete inserts examples.
%
%\key{the title to be shown}{\#+TITLE:}
%\key{the author}{\#+AUTHOR:}
%\key{authors email address}{\#+EMAIL:}
%\key{language code for html}{\#+LANGUAGE:}
%\key{free text description of file}{\#+TEXT:}
%\key{... which can carry over multiple lines}{\#+TEXT:}
%\key{settings for the export process}{\#+OPTIONS:}

%\key{set number of headline levels for export}{H:2}
%\key{turn on/off section numbers}{num:t}
%\key{turn on/off table of contents}{toc:t}
%\key{turn on/off linebreak preservation}{\\n:nil}
%\key{turn on/off quoted html tags}{@:t}
%\key{turn on/off fixed width sections}{::t}
%\key{turn on/off tables}{|:t}
%\key{turn on/off \TeX\ syntax for sub/super-scripts}{\^{}:t}
%\key{turn on/off emphasised text}{*:nil}
%\key{turn on/off \TeX\ macros}{TeX:t}

{\bf Comments: Text not being exported}

Lines starting with \kbd{\#} and subtrees starting with COMMENT are
never exported.

\key{toggle COMMENT keyword on entry}{C-c ;}

\section{Dynamic Blocks}

\key{update dynamic block at point}{C-c C-x C-u}
\metax{update all dynamic blocks}{C-u C-c C-x C-u}

\section{Notes}
$^1$ This is only a suggestion for a binding of this command.  Choose
your own key as shown under INSTALLATION.

$^2$ After changing a \kbd{\#+KEYWORD} or \kbd{<<<target>>>} line,
press \kbd{C-c C-c} with the cursor still in the line to update.

$^3$ Keybinding affected by {\tt org-support-shift-select} and
 {\tt org-replace-disputed-keys}.

\copyrightnotice

\bye

% Local variables:
% compile-command: "tex refcard"
% End:

% arch-tag: 139f6750-5cfc-49ca-92b5-237fe5795290
