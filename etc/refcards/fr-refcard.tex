% Reference Card for GNU Emacs

% Copyright (C) 1987, 1993, 1996-1997, 2001-2012
%   Free Software Foundation, Inc.

% Author: Stephen Gildea <gildea@stop.mail-abuse.org>
% French translation: Eric Jacoboni
%	Micha\"el Cadilhac

% This file is part of GNU Emacs.

% GNU Emacs is free software: you can redistribute it and/or modify
% it under the terms of the GNU General Public License as published by
% the Free Software Foundation, either version 3 of the License, or
% (at your option) any later version.

% GNU Emacs is distributed in the hope that it will be useful,
% but WITHOUT ANY WARRANTY; without even the implied warranty of
% MERCHANTABILITY or FITNESS FOR A PARTICULAR PURPOSE.  See the
% GNU General Public License for more details.

% You should have received a copy of the GNU General Public License
% along with GNU Emacs.  If not, see <http://www.gnu.org/licenses/>.


% This file is intended to be processed by plain TeX (TeX82).
%
% The final reference card has six columns, three on each side.
% This file can be used to produce it in any of three ways:
% 1 column per page
%    produces six separate pages, each of which needs to be reduced to 80%.
%    This gives the best resolution.
% 2 columns per page
%    produces three already-reduced pages.
%    You will still need to cut and paste.
% 3 columns per page
%    produces two pages which must be printed sideways to make a
%    ready-to-use 8.5 x 11 inch reference card.
%    For this you need a dvi device driver that can print sideways.
% Which mode to use is controlled by setting \columnsperpage.
%
% To compile and print this document:
% tex fr-refcard.tex
% dvips -t landscape fr-refcard.dvi


%**start of header

% This file can be printed with 1, 2, or 3 columns per page.
% Specify how many you want here.
\newcount\columnsperpage
\columnsperpage=3

% PDF output layout.  0 for A4, 1 for Letter (US), a `l' is added for
% a landscape layout.
\input pdflayout.sty
\pdflayout=(0l)

\def\versionemacs{24}           % version of Emacs this is for
\def\year{2012}                 % latest copyright year

% Nothing else needs to be changed below this line.

\def\shortcopyrightnotice{\vskip 1ex plus 2 fill
  \centerline{\small \copyright\ \year\ Free Software Foundation, Inc.
  Permissions au dos.}}

\def\copyrightnotice{
\vskip 1ex plus 2 fill\begingroup\small
\centerline{Copyright \copyright\ \year\ Free Software Foundation, Inc.}
\centerline{Pour GNU Emacs version \versionemacs}
\centerline{Conception de Stephen Gildea}
\centerline{Traduction fran\c{c}aise de Micha\"el Cadilhac}
% previously: Eric Jacoboni

Vous pouvez faire et distribuer des copies de cette carte, modifi�e ou
non, pourvu que la note de copyright et cette note de permission
soient conserv\'ees sur toutes les copies.

Pour des copies du manuel GNU Emacs:

{\tt http://www.gnu.org/software/emacs/\#Manuals}
\endgroup}

% make \bye not \outer so that the \def\bye in the \else clause below
% can be scanned without complaint.
\def\bye{\par\vfill\supereject\end}

\newdimen\intercolumnskip       %horizontal space between columns
\newbox\columna                 %boxes to hold columns already built
\newbox\columnb

\def\ncolumns{\the\columnsperpage}

\message{[\ncolumns\space
  column\if 1\ncolumns\else s\fi\space per page]}

\def\scaledmag#1{ scaled \magstep #1}

% This multi-way format was designed by Stephen Gildea October 1986.
% Note that the 1-column format is fontfamily-independent.
\if 1\ncolumns                  %one-column format uses normal size
  \hsize 4in
  \vsize 10in
  \voffset -.7in
  \font\titlefont=\fontname\tenbf \scaledmag3
  \font\headingfont=\fontname\tenbf \scaledmag2
  \font\smallfont=\fontname\sevenrm
  \font\smallsy=\fontname\sevensy

  \footline{\hss\folio}
  \def\makefootline{\baselineskip10pt\hsize6.5in\line{\the\footline}}
\else                           %2 or 3 columns uses prereduced size
  \hsize 3.2in
  \vsize 7.95in
%  \hoffset -.75in
  \hoffset -.49in
%  \voffset -.745in
  \voffset -.84in
  \font\titlefont=cmbx10 \scaledmag2
  \font\headingfont=cmbx10 \scaledmag1
  \font\smallfont=cmr6
  \font\smallsy=cmsy6
  \font\eightrm=cmr8
  \font\eightbf=cmbx8
  \font\eightit=cmti8
  \font\eighttt=cmtt8
  \font\eightmi=cmmi8
  \font\eightsy=cmsy8
  \textfont0=\eightrm
  \textfont1=\eightmi
  \textfont2=\eightsy
  \def\rm{\eightrm}
  \def\bf{\eightbf}
  \def\it{\eightit}
  \def\tt{\eighttt}
  \normalbaselineskip=.8\normalbaselineskip
  \normallineskip=.8\normallineskip
  \normallineskiplimit=.8\normallineskiplimit
  \normalbaselines\rm           %make definitions take effect

  \if 2\ncolumns
    \let\maxcolumn=b
    \footline{\hss\rm\folio\hss}
    \def\makefootline{\vskip 2in \hsize=6.86in\line{\the\footline}}
  \else \if 3\ncolumns
    \let\maxcolumn=c
    \nopagenumbers
  \else
    \errhelp{You must set \columnsperpage equal to 1, 2, or 3.}
    \errmessage{Illegal number of columns per page}
  \fi\fi

  \intercolumnskip=.46in
  \def\abc{a}
  \output={%                    %see The TeXbook page 257
      % This next line is useful when designing the layout.
      %\immediate\write16{Column \folio\abc\space starts with \firstmark}
      \if \maxcolumn\abc \multicolumnformat \global\def\abc{a}
      \else\if a\abc
        \global\setbox\columna\columnbox \global\def\abc{b}
        %% in case we never use \columnb (two-column mode)
        \global\setbox\columnb\hbox to -\intercolumnskip{}
      \else
        \global\setbox\columnb\columnbox \global\def\abc{c}\fi\fi}
  \def\multicolumnformat{\shipout\vbox{\makeheadline
      \hbox{\box\columna\hskip\intercolumnskip
        \box\columnb\hskip\intercolumnskip\columnbox}
      \makefootline}\advancepageno}
  \def\columnbox{\leftline{\pagebody}}

  \def\bye{\par\vfill\supereject
    \if a\abc \else\null\vfill\eject\fi
    \if a\abc \else\null\vfill\eject\fi
    \end}
\fi

% we won't be using math mode much, so redefine some of the characters
% we might want to talk about
\catcode`\^=12
\catcode`\_=12

\chardef\\=`\\
\chardef\{=`\{
\chardef\}=`\}

\hyphenation{mini-tam-pon}

\parindent 0pt
\parskip 1ex plus .5ex minus .5ex

\def\small{\smallfont\textfont2=\smallsy\baselineskip=.8\baselineskip}

% newcolumn - force a new column.  Use sparingly, probably only for
% the first column of a page, which should have a title anyway.
\outer\def\newcolumn{\vfill\eject}

% title - page title.  Argument is title text.
\outer\def\title#1{{\titlefont\centerline{#1}}\vskip 1ex plus .5ex}

% section - new major section.  Argument is section name.
\outer\def\section#1{\par\filbreak
  \vskip 2ex plus 1.5ex minus 2.5ex {\headingfont #1}\mark{#1}%
  \vskip 1.5ex plus 1ex minus 1.5ex}

\newdimen\keyindent

% beginindentedkeys...endindentedkeys - key definitions will be
% indented, but running text, typically used as headings to group
% definitions, will not.
\def\beginindentedkeys{\keyindent=1em}
\def\endindentedkeys{\keyindent=0em}
\endindentedkeys

% paralign - begin paragraph containing an alignment.
% If an \halign is entered while in vertical mode, a parskip is never
% inserted.  Using \paralign instead of \halign solves this problem.
\def\paralign{\vskip\parskip\halign}

% \<...> - surrounds a variable name in a code example
\def\<#1>{{\it #1\/}}

% kbd - argument is characters typed literally.  Like the Texinfo command.
\def\kbd#1{{\tt#1}\null}        %\null so not an abbrev even if period follows

% beginexample...endexample - surrounds literal text, such a code example.
% typeset in a typewriter font with line breaks preserved
\def\beginexample{\par\leavevmode\begingroup
  \obeylines\obeyspaces\parskip0pt\tt\tolerance=10000}
{\obeyspaces\global\let =\ }
\def\endexample{\endgroup}

% key - definition of a key.
% \key{description of key}{key-name}
% prints the description left-justified, and the key-name in a \kbd
% form near the right margin.
\def\key#1#2{\leavevmode\hbox to \hsize{\vtop
  {\hsize=.75\hsize\rightskip=1em
  \hskip\keyindent\relax#1}\kbd{#2}\hfil}}

\newbox\metaxbox
\setbox\metaxbox\hbox{\kbd{M-x }}
\newdimen\metaxwidth
\metaxwidth=\wd\metaxbox

% metax - definition of a M-x command.
% \metax{description of command}{M-x command-name}
% Tries to justify the beginning of the command name at the same place
% as \key starts the key name.  (The "M-x " sticks out to the left.)
% Note: was \hsize=.74, but changed to avoid overflow in some places.
\def\metax#1#2{\leavevmode\hbox to \hsize{\hbox to .75\hsize
  {\hskip\keyindent\relax#1\hfil}%
  \hskip -\metaxwidth minus 1fil
  \kbd{#2}\hfil}}

% threecol - like "key" but with two key names.
% for example, one for doing the action backward, and one for forward.
\def\threecol#1#2#3{\hskip\keyindent\relax#1\hfil&\kbd{#2}\hfil\quad
  &\kbd{#3}\hfil\quad\cr}

%**end of header


\title{Carte de r\'ef\'erence de GNU Emacs}

\centerline{(pour la version \versionemacs)}

\section{Lancer Emacs}

Pour lancer GNU Emacs \versionemacs, tapez juste son nom : \kbd{emacs}.

\section{Quitter Emacs}

\key{Suspendre Emacs (ou l'iconifier sous X)}{C-z}
\key{Quitter d\'efinitivement Emacs}{C-x C-c}

\section{Fichiers}

\key{{\bf Lire} un fichier}{C-x C-f}
\key{{\bf Sauvegarder} un fichier}{C-x C-s}
\key{Sauvegarder {\bf tous} les fichiers}{C-x s}
\key{{\bf Ins\'erer} un fichier sous le point (curseur)}{C-x i}
\key{Remplacer le fichier par un autre fichier}{C-x C-v}
\key{Sauvegarder sous un autre nom de fichier}{C-x C-w}
\key{Basculer en mode lecture seule}{C-x C-q}

\section{Obtenir de l'aide}

Le syst\`eme d'aide est simple. Faites \kbd{C-h} (ou \kbd{F1}) et
suivez les instructions. Si vous d\'ebutez, faites \kbd{C-h t} pour un
{\bf didacticiel}.

\key{Supprimer la fen\^etre d'aide}{C-x 1}
\key{Faire d\'efiler la fen\^etre d'aide}{C-M-v}

\key{Rechercher des commandes selon une cha\^\i{}ne}{C-h a}
\key{D\'ecrire la fonction associ\'ee \`a une touche}{C-h k}
\key{D\'ecrire une fonction}{C-h f}
\key{Obtenir des informations relatives au mode}{C-h m}

\section{R\'ecup\'eration sur erreur}

\key{{\bf Abandonner} une commande}{C-g}
\metax{{\bf R\'ecup\'erer} les fichiers apr\`es un crash}{M-x recover-session}
\metax{{\bf Annuler} une modification}{C-x u, C-_ {\rm ou} C-/}
\metax{Annuler toutes les modifications}{M-x revert-buffer}
\key{R\'eafficher un \'ecran perturb\'e}{C-l}

\section{Recherche incr\'ementale}

\key{Rechercher en avant}{C-s}
\key{Rechercher en arri\`ere}{C-r}
\key{Rechercher en avant (expression rationnelle)}{C-M-s}
\key{Rechercher en arri\`ere (expression rationnelle)}{C-M-r}

\key{Utiliser la cha\^\i{}ne de recherche pr\'ec\'edente}{M-p}
\key{Utiliser la cha\^\i{}ne de recherche suivante}{M-n}
\key{Quitter la recherche incr\'ementale}{RET}
\key{Annuler l'effet du dernier caract\`ere}{DEL}
\key{Annuler la recherche en cours}{C-g}

Refaites \kbd{C-s} ou \kbd{C-r} pour r\'ep\'eter la recherche dans une
des directions.
En cours de recherche, \kbd{C-g} efface les derniers caract\`eres et
ne conserve que le pr\'efixe d\'ej\`a trouv\'e.

\shortcopyrightnotice

\section{D\'eplacements}

\paralign to \hsize{#\tabskip=10pt plus 1 fil&#\tabskip=0pt&#\cr
\threecol{{\bf Objet sur lequel se d\'eplacer}}{{\bf En
    arri\`ere}}{{\bf En avant}}
\threecol{Caract\`ere}{C-b}{C-f}
\threecol{Mot}{M-b}{M-f}
\threecol{Ligne}{C-p}{C-n}
\threecol{Aller en d\'ebut/fin de la ligne}{C-a}{C-e}
\threecol{Phrase}{M-a}{M-e}
\threecol{Paragraphe}{M-\{}{M-\}}
\threecol{Page}{C-x [}{C-x ]}
\threecol{S-expression}{C-M-b}{C-M-f}
\threecol{Fonction}{C-M-a}{C-M-e}
\threecol{Aller en d\'ebut/fin du tampon}{M-<}{M->}
}

\key{Passer \`a l'\'ecran suivant}{C-v}
\key{Passer \`a l'\'ecran pr\'ec\'edent}{M-v}
\key{Faire d\'efiler l'\'ecran vers la gauche}{C-x <}
\key{Faire d\'efiler l'\'ecran vers la droite}{C-x >}
\key{Placer la ligne courante au centre de l'\'ecran}{C-u C-l}

\section{D\'etruire et supprimer}

\paralign to \hsize{#\tabskip=10pt plus 1 fil&#\tabskip=0pt&#\cr
\threecol{{\bf Objet \`a supprimer}}{{\bf En arri\`ere}}{{\bf En avant}}
\threecol{Caract\`ere (suppression)}{DEL}{C-d}
\threecol{Mot}{M-DEL}{M-d}
\threecol{Ligne (jusqu'au d\'ebut/fin)}{M-0 C-k}{C-k}
\threecol{Phrase}{C-x DEL}{M-k}
\threecol{S-expression}{M-- C-M-k}{C-M-k}
}

\key{D\'etruire une {\bf r\'egion}}{C-w}
\key{Copier une r\'egion dans le {\it kill ring}}{M-w}
\key{D\'etruire jusqu'\`a l'occurrence suivante de {\it car}}{M-z {\it car}}

\key{R\'ecup\'erer la derni\`ere r\'egion d\'etruite}{C-y}
\key{R\'ecup\'erer la r\'egion d\'etruite pr\'ec\'edente}{M-y}

\section{Marquer}

\key{Placer la marque au point}{C-@ {\rm ou} C-SPC}
\key{\'Echanger le point et la marque}{C-x C-x}

\key{Placer la marque un {\bf mot} plus loin}{M-@}
\key{Marquer le {\bf paragraphe}}{M-h}
\key{Marquer la {\bf page}}{C-x C-p}
\key{Marquer la {\bf s-expression}}{C-M-@}
\key{Marquer la {\bf fonction}}{C-M-h}
\key{Marquer tout le {\bf tampon}}{C-x h}

\section{Remplacement interactif}

\key{Remplacer une cha\^\i{}ne de texte}{M-\%}
% query-replace-regexp est liee a C-M-% mais on ne peut pas le
% taper dans une console.
\metax{\hskip 10pt \`a l'aide d'expr. rationnelles}{M-x query-replace-regexp}

R\'eponses possibles pour chaque occurrence dans le mode de
remplacement interactif :

\key{{\bf Remplacer} celle-l\`a, passer \`a la suivante}{SPC}
\key{Remplacer celle-l\`a, rester l\`a}{,}
\key{{\bf Passer} \`a la suivante sans remplacer}{DEL}
\key{Remplacer toutes les occurrences suivantes}{!}
\key{{\bf Revenir} \`a l'occurrence pr\'ec\'edente}{^}
\key{{\bf Quitter} le remplacement interactif}{RET}
\key{{\bf \'Editer} avant de reprendre (\kbd{C-M-c} : sortir)}{C-r}

\section{Fen\^etres multiples}

Lorsqu'il y a deux commandes, la seconde est celle qui concerne non
pas les fen\^etres mais les cadres.

{\setbox0=\hbox{\kbd{0}}\advance\hsize by 0\wd0
\paralign to \hsize{#\tabskip=10pt plus 1 fil&#\tabskip=0pt&#\cr
\threecol{Supprimer toutes les autres fen\^etres}{C-x 1\ \ \ \ }{C-x 5 1}
\threecol{Supprimer cette fen\^etre}{C-x 0\ \ \ \ }{C-x 5 0}
\threecol{Diviser la fen\^etre horizontalement}{C-x 2\ \ \ \ }{C-x 5 2}
}}
\key{Diviser la fen\^etre verticalement}{C-x 3}

\key{Faire d\'efiler l'autre fen\^etre}{C-M-v}

%% tabskip reduced from 10 to 3pt to fit on letterpaper.
{\setbox0=\hbox{\kbd{0}}\advance\hsize by 2\wd0
\paralign to \hsize{#\tabskip=3pt plus 1 fil&#\tabskip=0pt&#\cr
\threecol{S\'electionner une autre fen\^etre}{C-x o}{C-x 5 o}

\threecol{Choisir un tampon (autre fen\^etre)}{C-x 4 b}{C-x 5 b}
\threecol{Afficher un tampon (autre fen\^etre)}{C-x 4 C-o}{C-x 5 C-o}
\threecol{Lire un fichier (autre fen\^etre)}{C-x 4 f}{C-x 5 f}
\threecol{\hskip 10pt en lecture seule}{C-x 4 r}{C-x 5 r}
\threecol{Lancer Dired (autre fen\^etre)}{C-x 4 d}{C-x 5 d}
\threecol{Trouver un tag (autre fen\^etre)}{C-x 4 .}{C-x 5 .}
}}

\key{Agrandir la fen\^etre verticalement}{C-x ^}
\key{R\'eduire la fen\^etre horizontalement}{C-x \{}
\key{Agrandir la fen\^etre horizontalement}{C-x \}}

\section{Formater}

\key{Indenter la {\bf ligne} courante (selon le mode)}{TAB}
\key{Indenter la {\bf r\'egion} courante (selon le mode)}{C-M-\\}
\key{Indenter la {\bf s-expr.} courante (selon le mode)}{C-M-q}
\key{Indenter la r\'egion sur une colonne}{C-x TAB}
\key{Ins\'erer un retour \`a la ligne apr\`es le point}{C-o}
\key{D\'eplacer le reste de la ligne vers le bas}{C-M-o}
\key{Supprimer les lignes vierges autour du point}{C-x C-o}
\key{Joindre \`a la ligne pr\'ec\'edente (suiv. avec {\it arg\/})}{M-^}
\key{Supprimer tous les espaces autour du point}{M-\\}
\key{Mettre exactement une espace au point}{M-SPC}

\key{Formater le paragraphe}{M-q}
\key{Placer la marge droite \`a {\it arg\/} colonnes}{C-u {\it arg\/} C-x f}
\key{D\'efinir le pr\'efixe des lignes}{C-x .}

\key{D\'efinir la fonte}{M-o}

\section{Modifier la casse}

\key{Mettre le mot en capitales}{M-u}
\key{Mettre le mot en minuscules}{M-l}
\key{Mettre une majuscule au mot}{M-c}

\key{Mettre la r\'egion en capitales}{C-x C-u}
\key{Mettre la r\'egion en minuscules}{C-x C-l}

\section{Le mini-tampon}

Dans le mini-tampon :

\key{Compl\'eter autant que possible}{TAB}
\key{Compl\'eter un mot}{SPC}
\key{Compl\'eter et ex\'ecuter}{RET}
\key{Montrer les compl\`etements possibles}{?}
\key{Utiliser l'entr\'ee pr\'ec\'edente du mini-tampon}{M-p}
\key{Utiliser l'entr\'ee suivante du mini-tampon}{M-n}
\key{Rechercher en arri\`ere dans l'historique}{M-r}
\key{Rechercher en avant  dans l'historique}{M-s}
\key{Quitter en annulant la commande}{C-g}

Faites \kbd{C-x ESC ESC} pour \'editer et r\'ep\'eter la derni\`ere
commande ayant utilis\'e le mini-tampon. Faites \kbd{F10} pour
utiliser la barre de menu sur un terminal en utilisant le mini-tampon.

\newcolumn
\title{Carte de r\'ef\'erence de GNU Emacs}

\section{Tampons}

\key{Choisir un autre tampon}{C-x b}
\key{Lister tous les tampons}{C-x C-b}
\key{Supprimer un tampon}{C-x k}

\section{Transposer}

\key{Transposer des {\bf caract\`eres}}{C-t}
\key{Transposer des {\bf mots}}{M-t}
\key{Transposer des {\bf lignes}}{C-x C-t}
\key{Transposer des {\bf s-expressions}}{C-M-t}

\section{V\'erifier l'orthographe}

\key{V\'erifier l'orthographe du mot courant}{M-\$}
\metax{V\'erifier l'orthographe d'une r\'egion}{M-x ispell-region}
\metax{V\'erifier l'orthographe de tout le tampon}{M-x ispell-buffer}

\section{Tags}

\key{Trouver un tag (une d\'efinition)}{M-.}
\key{Passer \`a l'occurrence suivante du tag}{C-u M-.}
\metax{Sp\'ecifier un autre fichier de tags}{M-x visit-tags-table}

\metax{Rechercher dans tous les fichiers des tags}{M-x tags-search}

\metax{Remplacer dans tous les fichiers}{M-x tags-query-replace}
\key{Continuer la recherche ou le remplacement}{M-,}

\section{Shell}

\key{Ex\'ecuter une commande shell}{M-!}
\key{Lancer une commande shell sur la r\'egion}{M-|}
\key{Filtrer la r\'egion avec une commande shell}{C-u M-|}
\key{Lancer un shell dans la fen\^etre {\tt *shell*}}{M-x shell}

\section{Rectangles}

\key{Copier le rectangle dans un registre}{C-x r r}
\key{D\'etruire le rectangle}{C-x r k}
\key{R\'ecup\'erer le rectangle}{C-x r y}
\key{D\'ecaler le rectangle \`a droite}{C-x r o}
\key{Vider le rectangle}{C-x r c}
\key{Pr\'efixer chaque ligne du rectangle}{C-x r t}

\section{Abr\'eviations}

\key{Ajouter une abr\'eviation globale}{C-x a g}
\key{Ajouter une abr\'eviation locale au mode}{C-x a l}
\key{Ajouter une expansion globale}{C-x a i g}
\key{Ajouter une expansion locale au mode}{C-x a i l}
\key{Faire une expansion explicite de l'abr\'eviation}{C-x a e}

\key{Faire une expansion du mot pr\'ec\'edent}{M-/}

\section{Expressions rationnelles}

\key{Un caract\`ere quelconque, sauf fin de ligne}{. {\rm(point)}}
\key{Z\'ero r\'ep\'etition ou plus}{*}
\key{Une r\'ep\'etition ou plus}{+}
\key{Z\'ero ou une r\'ep\'etition}{?}
\key{\'Echapper le caract\`ere sp\'ecial {\it c\/}}{\\{\it c}}
\key{Alternative (``ou'' non exclusif)}{\\|}
\key{Regroupement}{\\( {\rm$\ldots$} \\)}
\key{Reprendre le texte du {\it n\/}-i\`eme groupement}{\\{\it n}}
\key{Limite de mot}{\\b}
\key{Non limite de mot}{\\B}

%% tabskip reduced from 10 to 5pt for letterpaper.
\paralign to \hsize{#\tabskip=5pt plus 1 fil&#\tabskip=0pt&#\cr
\threecol{{\bf Objet}}{{\bf D\'ebut}}{{\bf Fin}}
\threecol{Ligne}{^}{\$}
\threecol{Mot}{\\<}{\\>}
\threecol{Tampon}{\\`}{\\'}

\threecol{{\bf Classe de caract\`ere}}{{\bf Correspond}}%
{{\bf Compl\'ement}}
\threecol{Ensemble explicite}{[ {\rm$\ldots$} ]}{[^ {\rm$\ldots$} ]}
\threecol{Caract\`ere de mot}{\\w}{\\W}
\threecol{Caract\`ere avec la syntaxe {\it c}}{\\s{\it c}}{\\S{\it c}}
}

\section{Jeux de caract\`eres internationaux}

\key{Pr\'eciser la langue principale}{C-x RET l}
\metax{Lister les m\'ethodes de saisie}{M-x list-input-methods}
\key{Activer/d\'esactiver la m\'ethode de saisie}{C-\\}
\key{Choisir le codage pour la commande suivante}{C-x RET c}
\metax{Lister les codages}{M-x list-coding-systems}
\metax{Choisir le codage pr\'ef\'er\'e}{M-x prefer-coding-system}

\section{Info}

\key{Lire une documentation Info}{C-h i}
\key{Rechercher une fonction/variable dans Info}{C-h S}
\beginindentedkeys

Se d\'eplacer dans un n\oe{}ud :

\key{Page suivante}{SPC}
\key{Page pr\'ec\'edente}{DEL}
\key{D\'ebut du n\oe{}ud}{. {\rm (point)}}

Navigation entre n\oe{}uds :

\key{N\oe{}ud {\bf suivant}}{n}
\key{N\oe{}ud {\bf pr\'ec\'edent}}{p}
\key{{\bf Remonter} d'un niveau}{u}
\key{Choisir un sujet du menu par son nom}{m}
\key{Choisir le {\it n\/}-i\`eme sujet (1--9)}{{\it n}}
\key{Suivre une r\'ef\'erence crois\'ee (retour avec \kbd{l})}{f}
\key{Revenir au dernier n\oe{}ud visit\'e}{l}
\key{Aller au sommaire Info}{d}
\key{Aller au n\oe{}ud le plus haut du manuel}{t}
\key{Aller sur un n\oe{}ud par son nom}{g}

Autres :

\key{Aller au {\bf didacticiel} Info}{h}
\key{Rechercher un sujet dans l'index}{i}
\key{Rechercher un n\oe{}ud par expr. rationnelle}{s}
\key{{\bf Quitter} Info}{q}

\endindentedkeys

\section{Registres}

\key{Sauver la r\'egion dans un registre}{C-x r s}
\key{Ins\'erer le contenu d'un registre}{C-x r i}

\key{Sauver la valeur du point dans un registre}{C-x r SPC}
\key{Aller au point sauv\'e dans un registre}{C-x r j}

\section{Macros clavier}

\key{{\bf Lancer} la d\'efinition d'une macro clavier}{C-x (}
\key{{\bf Terminer} la d\'efinition d'une macro clavier}{C-x )}
\key{{\bf Ex\'ecuter} la derni\`ere macro clavier d\'efinie}{C-x e}
\key{Faire un ajout \`a la derni\`ere macro clavier}{C-u C-x (}
\metax{Nommer la derni\`ere macro clavier}{M-x name-last-kbd-macro}
\metax{En ins\'erer une d\'efinition Lisp}{M-x insert-kbd-macro}

\section{Commandes de gestion d'Emacs Lisp}

\key{\'Evaluer la {\bf s-expression} avant le point}{C-x C-e}
\key{\'Evaluer la {\bf defun} courante}{C-M-x}
\metax{\'Evaluer la {\bf r\'egion}}{M-x eval-region}
\key{Lire et \'evaluer dans le mini-tampon}{M-:}
\metax{Charger depuis un r\'epertoire standard}{M-x load-library}

\section{Personnalisation simple}

\metax{Personnaliser les variables et les fontes}{M-x customize}

% The intended audience here is the person who wants to make simple
% customizations and knows Lisp syntax.

Exemples de d\'efinition globale de touches en Emacs Lisp :

\beginexample%
(global-set-key (kbd "C-c g") 'search-forward)
(global-set-key (kbd "M-\#") 'query-replace-regexp)
\endexample

\section{\'Ecriture de commandes}

\beginexample%
(defun \<nom-commande> (\<args>)
  "\<documentation>"
  (interactive "\<template>")
  \<body>)
\endexample

Exemple :

\beginexample%
(defun cette-ligne-en-haut-de-la-fenetre (line)
  "Positionne la ligne courante en haut de la fen\^etre.
Avec ARG, place le point sur la ligne ARG."
  (interactive "P")
  (recenter (if (null line)
                0
              (prefix-numeric-value line))))
\endexample

La sp\'ecification \kbd{interactive} indique comment lire
interactivement les param\`etres. Faites \kbd{C-h f interactive} pour
plus de pr\'ecisions.

\copyrightnotice

\bye

% Local variables:
% compile-command: "pdftex fr-refcard"
% End:

