%&tex
%
% Title:  GNU Emacs Survival Card
% Author: Wlodek Bzyl <matwb@univ.gda.pl>
%
%**start of header

% User interface is `plain.tex' and macros described below
%
% \title{CARD TITLE}{for version 21}
% \section{NAME}
% optional paragraphs separated with \askip amount of vertical space
% \key{KEY-NAME} description of key or
% \mkey{M-x LONG-LISP-NAME} description of Elisp function
%
% \kbd{ARG} -- argument is typed literally

\def\plainfmtname{plain}
\ifx\fmtname\plainfmtname
\else
  \errmessage{This file requires `plain' format to be typeset correctly}
  \endinput
\fi

% PDF output layout.  0 for A4, 1 for letter (US), a `l' is added for
% a landscape layout.

\input pdflayout.sty
\pdflayout=(1)

% Copyright (C) 2000, 2001, 2002, 2003, 2004, 2005,
%   2006, 2007, 2008, 2009  Free Software Foundation, Inc.

% This file is part of GNU Emacs.

% GNU Emacs is free software: you can redistribute it and/or modify
% it under the terms of the GNU General Public License as published by
% the Free Software Foundation, either version 3 of the License, or
% (at your option) any later version.

% GNU Emacs is distributed in the hope that it will be useful,
% but WITHOUT ANY WARRANTY; without even the implied warranty of
% MERCHANTABILITY or FITNESS FOR A PARTICULAR PURPOSE.  See the
% GNU General Public License for more details.

% You should have received a copy of the GNU General Public License
% along with GNU Emacs.  If not, see <http://www.gnu.org/licenses/>.

\def\versionnumber{1.0}
\def\versionemacs{21}
\def\year{2008}                 % latest copyright year

\def\copyrightnotice{\penalty-1\vfill
  \vbox{\smallfont\baselineskip=0.8\baselineskip\raggedcenter
    Copyright \year\ Free Software Foundation, Inc.\break
    Version \versionnumber{} for GNU Emacs \versionemacs, April 2000\break
    Project W{\l}odek Bzyl (matwb@univ.gda.pl)

    Permission is granted to make and distribute copies of
    this card provided the copyright notice and this permission notice
    are preserved on all copies.\par}}

\hsize 3.2in
\vsize 7.95in
\font\titlefont=cmss10 scaled 1200
\font\headingfont=cmss10
\font\smallfont=cmr6
\font\smallsy=cmsy6
\font\eightrm=cmr8
\font\eightbf=cmbx8
\font\eightit=cmti8
\font\eighttt=cmtt8
\font\eightmi=cmmi8
\font\eightsy=cmsy8
\font\eightss=cmss8
\textfont0=\eightrm
\textfont1=\eightmi
\textfont2=\eightsy
\def\rm{\eightrm} \rm
\def\bf{\eightbf}
\def\it{\eightit}
\def\tt{\eighttt}
\def\ss{\eightss}
\baselineskip=0.8\baselineskip

\newdimen\intercolumnskip % horizontal space between columns
\intercolumnskip=0.5in

% The TeXbook, p. 257
\let\lr=L \newbox\leftcolumn
\output={\if L\lr
    \global\setbox\leftcolumn\columnbox \global\let\lr=R
  \else
       \doubleformat \global\let\lr=L\fi}
\def\doubleformat{\shipout\vbox{\makeheadline
    \leftline{\box\leftcolumn\hskip\intercolumnskip\columnbox}
    \makefootline}
  \advancepageno}
\def\columnbox{\leftline{\pagebody}}

\def\newcolumn{\vfil\eject}

\def\bye{\par\vfil\supereject
  \if R\lr \null\vfil\eject\fi
  \end}

\outer\def\title#1#2{{\titlefont\centerline{#1}}\vskip 1ex plus 0.5ex
   \centerline{\ss#2}
   \vskip2\baselineskip}

\outer\def\section#1{\filbreak
  \bskip
  \leftline{\headingfont #1}
  \askip}
\def\bskip{\vskip 2.5ex plus 0.25ex }
\def\askip{\vskip 0.75ex plus 0.25ex}

\newdimen\defwidth \defwidth=0.25\hsize
\def\hang{\hangindent\defwidth}

\def\textindent#1{\noindent\llap{\hbox to \defwidth{\tt#1\hfil}}\ignorespaces}
\def\key{\par\hangafter=0\hang\textindent}

\def\mtextindent#1{\noindent\hbox{\tt#1\quad}\ignorespaces}
\def\mkey{\par\hangafter=1\hang\mtextindent}

\def\kbd#{\bgroup\tt \let\next= }

\newdimen\raggedstretch
\newskip\raggedparfill \raggedparfill=0pt plus 1fil
\def\nohyphens
   {\hyphenpenalty10000\exhyphenpenalty10000\pretolerance10000}
\def\raggedspaces
   {\spaceskip=0.3333em\relax
    \xspaceskip=0.5em\relax}
\def\raggedright
   {\raggedstretch=6em
    \nohyphens
    \rightskip=0pt plus \raggedstretch
    \raggedspaces
    \parfillskip=\raggedparfill
    \relax}
\def\raggedcenter
   {\raggedstretch=6em
    \nohyphens
    \rightskip=0pt plus \raggedstretch
    \leftskip=\rightskip
    \raggedspaces
    \parfillskip=0pt
    \relax}

\chardef\\=`\\

\raggedright
\nopagenumbers
\parindent 0pt
\interlinepenalty=10000
\hoffset -0.2in
%\voffset 0.2in

%**end of header


\title{GNU\ \ Emacs\ \ Survival\ \ Card}{for version \versionemacs}

In the following, \kbd{C-z} means hit the `\kbd{z}' key while
holding down the {\it Ctrl}\ \ key. \kbd{M-z} means hit the
`\kbd{z}' key while hitting the {\it Meta\/} (labeled {\it Alt\/}
on some keyboards) or after hitting {\it Esc\/} key.

\section{Running Emacs}

To enter GNU Emacs, just type its name: \kbd{emacs}.
Emacs divides the frame into several areas:
  menu line,
  buffer area with the edited text,
  mode line describing the buffer in the window above it,
  and a minibuffer/echo area in the last line.
\askip
\key{C-x C-c} quit Emacs
\key{C-x C-f} edit file; this command uses the minibuffer to read
  the file name; use this to create new files by entering the name
  of the new file
\key{C-x C-s} save the file
\key{C-x k} kill a buffer
\key{C-g} in most context: cancel, stop, abort partially typed or
  executing command
\key{C-x u} undo

\section{Moving About}

\key{C-l} scroll current line to center of window
\key{C-x b} switch to another buffer
\key{M-<} move to beginning of buffer
\key{M->} move to end of buffer
\key{M-x goto-line} go to a given line number

\section{Multiple Windows}

\key{C-x 0} remove the current window from the display
\key{C-x 1} make active window the only window
\key{C-x 2} split window horizontally
\key{C-x 3} split window vertically
\key{C-x o} move to other window

\section{Regions}

Emacs defines a `region' as the space between the {\it mark\/} and
the {\it point}. A mark is set with \kbd{C-{\it space}}.
The point is at the cursor position.
\askip
\key{M-h} mark entire paragraph
\key{C-x h} mark entire buffer

\section{Killing and Copying}

\key{C-w} kill region
\key{M-w} copy region to kill-ring
\key{C-k} kill from the cursor all the way to the end of the line
\key{M-DEL} kill word
\key{C-y} yank back the last kill (\kbd{C-w C-y} combination could be
  used to move text around)
\key{M-y} replace last yank with previous kill

\section{Searching}

\key{C-s} search for a string
\key{C-r} search for a string backwards
\key{RET} quit searching
\key{M-C-s} regular expression search
\key{M-C-r} reverse regular expression search
\askip
Use \kbd{C-s} or \kbd{C-r} again to repeat the search in either direction.

\section{Tags}

Tags tables files record locations of function and
procedure definitions, global variables, data types and anything
else convenient. To create a tags table file, type
`{\tt etags} {\it input\_files}' as a shell command.
\askip
\key{M-.} find a definition
\key{C-u M-.} find next occurrence of definition
\key{M-*} pop back to where \kbd{M-.} was last invoked
\mkey{M-x tags-query-replace} run query-replace on all files
  recorded in tags table
\key{M-,} continue last tags search or query-replace

\section{Compiling}

\key{M-x compile} compile code in active window
\key{C-c C-c} go to the next compiler error, when in
  the compile window or
\key{C-x `} when in the window with source code

\section{Dired, the Directory Editor}

\key{C-x d} invoke Dired
\key{d} flag this file for deletion
\key{\~{}} flag all backup files for deletion
\key{u} remove deletion flag
\key{x} delete the files flagged for deletion
\key{C} copy file
\key{g} update the Dired buffer
\key{f} visit the file described on the current line
\key{s} switch between alphabetical date/time order

\section{Reading and Sending Mail}

\key{M-x rmail} start reading mail
\key{q} quit reading mail
\key{h} show headers
\key{d} mark the current message for deletion
\key{x} remove all messages marked for deletion

\key{C-x m} begin composing a message
\key{C-c C-c} send the message and switch to another buffer
\key{C-c C-f C-c} move to the `CC' header field, creating one
  if there is none

\section{Miscellaneous}

\key{M-q} fill paragraph
\key{M-/} expand previous word dynamically
\key{C-z} iconify (suspend) Emacs when running it under X or
  shell, respectively
\mkey{M-x revert-buffer} replace the text being edited with the
  text of the file on disk

\section{Query Replace}

\key{M-\%} interactively search and replace
\key{M-C-\%} using regular expressions
\askip
Valid responses in query-replace mode are
\askip
\key{SPC} replace this one, go on to next
\key{,} replace this one, don't move
\key{DEL} skip to next without replacing
\key{!} replace all remaining matches
\key{\^{}} back up to the previous match
\key{RET} exit query-replace
\key{C-r} enter recursive edit (\kbd{M-C-c} to exit)

\section{Regular Expressions}

\key{. {\rm(dot)}} any single character except a newline
\key{*} zero or more repeats
\key{+} one or more repeats
\key{?} zero or one repeat
\key{[$\ldots$]} denotes a class of character to match
\key{[\^{}$\ldots$]} negates the class

\key{\\{\it c}} quote characters otherwise having a special
  meaning in regular expressions

\key{$\ldots$\\|$\ldots$\\|$\ldots$} matches one of
  the alternatives (``or'')
\key{\\( $\ldots$ \\)} groups a series of pattern elements to
  a single element
\key{\\{\it n}} same text as {\it n\/}th group

\key{\^{}} matches at line beginning
\key{\$} matches at line end

\key{\\w} matches word-syntax character
\key{\\W} matches non-word-syntax character
\key{\\<} matches at word beginning
\key{\\>} matches at word end
\key{\\b} matches at word break
\key{\\B} matches at non-word break

\section{Registers}

\key{C-x r s} save region in register
\key{C-x r i} insert register contents into buffer

\key{C-x r SPC} save value of point in register
\key{C-x r j} jump to point saved in register

\section{Rectangles}

\key{C-x r r} copy rectangle to register
\key{C-x r k} kill rectangle
\key{C-x r y} yank rectangle
\key{C-x r t} prefix each line with a string

\key{C-x r o} open rectangle, shifting text right
\key{C-x r c} blank out rectangle

\section{Shells}

\key{M-x shell} start a shell within Emacs
\key{M-!} execute a shell command
\key{M-|} run a shell command on the region
\key{C-u M-|} filter region through a shell command

\section{Spelling Check}

\key{M-\$} check spelling of word at the cursor
\mkey{M-x ispell-region} check spelling of all words in region
\mkey{M-x ispell-buffer} check spelling of entire buffer

\section{International Character Sets}

\key{C-x RET C-\\} select and activate input method for
  the current buffer
\key{C-\\} enable or disable input method
\mkey{M-x list-input-methods} show all input methods
\mkey{M-x set-language-environment} specify principal language

\key{C-x RET c} set coding system for next command
\mkey{M-x find-file-literally} visit file with no conversion
  of any kind

\mkey{M-x list-coding-systems} show all coding systems
\mkey{M-x prefer-coding-system} choose preferred coding system

\section{Keyboard Macros}

\key{C-x (} start defining a keyboard macro
\key{C-x )} end keyboard macro definition
\key{C-x e} execute last-defined keyboard macro
\key{C-u C-x (} append to last keyboard macro
\mkey{M-x name-last-kbd-macro} name last keyboard macro

\section{Simple Customization}

\key{M-x customize} customize variables and faces

\section{Getting Help}

Emacs does command completion for you. Typing \kbd{M-x}
{\it tab\/} or {\it space\/} gives a list of Emacs commands.
\askip
\key{C-h} Emacs help
\key{C-h t} run the Emacs tutorial
\key{C-h i} enter Info, the documentation browser
\key{C-h a} show commands matching a string (apropos)
\key{C-h k} display documentation of the function invoked by
  keystroke
\askip
Emacs gets into different {\it modes}, each of which customizes
Emacs for editing text of a particular sort. The mode line
contains names of the current modes, in parentheses.
\askip
\key{C-h m} get mode-specific information

\copyrightnotice

\bye

% Local variables:
% compile-command: "pdftex survival"
% End:

% arch-tag: 4f9a0562-617b-4843-aee1-450c41d6b22c
