% Reference Card for Dired
% Copyright (C) 2000, 2001, 2002, 2003, 2004,
%   2005, 2006, 2007  Free Software Foundation, Inc.

% This file is part of GNU Emacs.

% GNU Emacs is free software; you can redistribute it and/or modify
% it under the terms of the GNU General Public License as published by
% the Free Software Foundation; either version 2, or (at your option)
% any later version.

% GNU Emacs is distributed in the hope that it will be useful,
% but WITHOUT ANY WARRANTY; without even the implied warranty of
% MERCHANTABILITY or FITNESS FOR A PARTICULAR PURPOSE.  See the
% GNU General Public License for more details.

% You should have received a copy of the GNU General Public License
% along with GNU Emacs; see the file COPYING.  If not, write to
% the Free Software Foundation, Inc., 51 Franklin Street, Fifth Floor,
% Boston, MA 02110-1301, USA.

% This file can be printed with 1, 2, or 3 columns per page (see below).
% Specify how many you want here.  Nothing else needs to be changed.
%**start of header

\newcount\columnsperpage

\columnsperpage=2

% This file is intended to be processed by plain TeX (TeX82).
% The reference card looks OK with 2 columns per page, portrait mode.
% I haven't tried it with 3 columns per page.

% This is a bit of a dirty hack on the GNU Emacs reference card
% to produce a Dired reference card instead.

% I (Evgeny Roubinchtein, eroubinc@u.washington.edu) put this together
% because I wanted a Dired reference card,
% but couldn't find anything on the 'net.
% Based mostly off Dired's describe-mode.


\def\versionnumber{0.1}
\def\year{2007}                 % latest copyright year
\def\version{May 2006\ v\versionnumber}

\def\shortcopyrightnotice{\vskip 1ex plus 2 fill
  \centerline{\small \copyright\ \year\ Free Software Foundation, Inc.
  Permissions on back.  v\versionnumber}}

\def\copyrightnotice{
\vskip 1ex plus 2 fill\begingroup\small
\centerline{Copyright \copyright\ \year\ Free Software Foundation, Inc.}
\centerline{designed by Stephen Gildea, \version}
\centerline{for GNU Emacs version 19 on Unix systems}
\centerline{Updated for Dired in May 2000 by Evgeny Roubinchtein}

Permission is granted to make and distribute copies of
this card provided the copyright notice and this permission notice
are preserved on all copies.

For copies of the GNU Emacs manual, write to the Free Software
Foundation, Inc., 51 Franklin Street, Fifth Floor, Boston,
MA 02110-1301, USA.

\endgroup}

% make \bye not \outer so that the \def\bye in the \else clause below
% can be scanned without complaint.
\def\bye{\par\vfill\supereject\end}

\newdimen\intercolumnskip       %horizontal space between columns
\newbox\columna                 %boxes to hold columns already built
\newbox\columnb

\def\ncolumns{\the\columnsperpage}

\message{[\ncolumns\space
  column\if 1\ncolumns\else s\fi\space per page]}

\def\scaledmag#1{ scaled \magstep #1}

% This multi-way format was designed by Stephen Gildea October 1986.
% Note that the 1-column format is fontfamily-independent.
\if 1\ncolumns                  %one-column format uses normal size
  \hsize 4in
  \vsize 10in
  \voffset -.7in
  \font\titlefont=\fontname\tenbf \scaledmag3
  \font\headingfont=\fontname\tenbf \scaledmag2
  \font\smallfont=\fontname\sevenrm
  \font\smallsy=\fontname\sevensy

  \footline{\hss\folio}
  \def\makefootline{\baselineskip10pt\hsize6.5in\line{\the\footline}}
\else                           %2 or 3 columns uses prereduced size
  \hsize 3.4in
  \vsize 9.2in
  \hoffset -.75in
  \voffset -.745in
  \font\titlefont=cmbx10 \scaledmag2
  \font\headingfont=cmbx10 \scaledmag1
  \font\smallfont=cmr6
  \font\smallsy=cmsy6
  \font\eightrm=cmr8
  \font\eightbf=cmbx8
  \font\eightit=cmti8
  \font\eighttt=cmtt8
  \font\eightmi=cmmi8
  \font\eightsy=cmsy8
  \textfont0=\eightrm
  \textfont1=\eightmi
  \textfont2=\eightsy
  \def\rm{\eightrm}
  \def\bf{\eightbf}
  \def\it{\eightit}
  \def\tt{\eighttt}
  \normalbaselineskip=.8\normalbaselineskip
  \normallineskip=.8\normallineskip
  \normallineskiplimit=.8\normallineskiplimit
  \normalbaselines\rm           %make definitions take effect

  \if 2\ncolumns
    \let\maxcolumn=b
    \footline{\hss\rm\folio\hss}
    \def\makefootline{\vskip 2in \hsize=6.86in\line{\the\footline}}
  \else \if 3\ncolumns
    \let\maxcolumn=c
    \nopagenumbers
  \else
    \errhelp{You must set \columnsperpage equal to 1, 2, or 3.}
    \errmessage{Illegal number of columns per page}
  \fi\fi

  \intercolumnskip=.46in
  \def\abc{a}
  \output={%                    %see The TeXbook page 257
      % This next line is useful when designing the layout.
      %\immediate\write16{Column \folio\abc\space starts with \firstmark}
      \if \maxcolumn\abc \multicolumnformat \global\def\abc{a}
      \else\if a\abc
        \global\setbox\columna\columnbox \global\def\abc{b}
        %% in case we never use \columnb (two-column mode)
        \global\setbox\columnb\hbox to -\intercolumnskip{}
      \else
        \global\setbox\columnb\columnbox \global\def\abc{c}\fi\fi}
  \def\multicolumnformat{\shipout\vbox{\makeheadline
      \hbox{\box\columna\hskip\intercolumnskip
        \box\columnb\hskip\intercolumnskip\columnbox}
      \makefootline}\advancepageno}
  \def\columnbox{\leftline{\pagebody}}

  \def\bye{\par\vfill\supereject
    \if a\abc \else\null\vfill\eject\fi
    \if a\abc \else\null\vfill\eject\fi
    \end}
\fi

% we won't be using math mode much, so redefine some of the characters
% we might want to talk about
\catcode`\^=12
\catcode`\_=12

\chardef\\=`\\
\chardef\{=`\{
\chardef\}=`\}

\hyphenation{mini-buf-fer}
\hyphenation{de-le-tion}

\parindent 0pt
\parskip 1ex plus .5ex minus .5ex

\def\small{\smallfont\textfont2=\smallsy\baselineskip=.8\baselineskip}

% newcolumn - force a new column.  Use sparingly, probably only for
% the first column of a page, which should have a title anyway.
\outer\def\newcolumn{\vfill\eject}

% title - page title.  Argument is title text.
\outer\def\title#1{{\titlefont\centerline{#1}}\vskip 1ex plus .5ex}

% section - new major section.  Argument is section name.
\outer\def\section#1{\par\filbreak
  \vskip 3ex plus 2ex minus 2ex {\headingfont #1}\mark{#1}%
  \vskip 2ex plus 1ex minus 1.5ex}

\newdimen\keyindent

% beginindentedkeys...endindentedkeys - key definitions will be
% indented, but running text, typically used as headings to group
% definitions, will not.
\def\beginindentedkeys{\keyindent=1em}
\def\endindentedkeys{\keyindent=0em}
\endindentedkeys

% paralign - begin paragraph containing an alignment.
% If an \halign is entered while in vertical mode, a parskip is never
% inserted.  Using \paralign instead of \halign solves this problem.
\def\paralign{\vskip\parskip\halign}

% \<...> - surrounds a variable name in a code example
\def\<#1>{{\it #1\/}}

% kbd - argument is characters typed literally.  Like the Texinfo command.
\def\kbd#1{{\tt#1}\null}        %\null so not an abbrev even if period follows

% beginexample...endexample - surrounds literal text, such a code example.
% typeset in a typewriter font with line breaks preserved
\def\beginexample{\par\leavevmode\begingroup
  \obeylines\obeyspaces\parskip0pt\tt}
{\obeyspaces\global\let =\ }
\def\endexample{\endgroup}

% key - definition of a key.
% \key{description of key}{key-name}
% prints the description left-justified, and the key-name in a \kbd
% form near the right margin.
\def\key#1#2{\leavevmode\hbox to \hsize{\vtop
  {\hsize=.75\hsize\rightskip=1em
  \hskip\keyindent\relax#1}\kbd{#2}\hfil}}

\newbox\metaxbox
\setbox\metaxbox\hbox{\kbd{M-x }}
\newdimen\metaxwidth
\metaxwidth=\wd\metaxbox

% metax - definition of a M-x command.
% \metax{description of command}{M-x command-name}
% Tries to justify the beginning of the command name at the same place
% as \key starts the key name.  (The "M-x " sticks out to the left.)
\def\metax#1#2{\leavevmode\hbox to \hsize{\hbox to .75\hsize
  {\hskip\keyindent\relax#1\hfil}%
  \hskip -\metaxwidth minus 1fil
  \kbd{#2}\hfil}}

% threecol - like "key" but with two key names.
% for example, one for doing the action backward, and one for forward.
\def\threecol#1#2#3{\hskip\keyindent\relax#1\hfil&\kbd{#2}\hfil\quad
  &\kbd{#3}\hfil\quad\cr}

% I cannot figure out how to make all dired-x
% commands fit on a page in two-column format
\def\dx{{\bf (DX)}}

\nopagenumbers

%**end of header


\title{Dired Reference Card}

\centerline{(based on Dired in GNU Emacs 21.x)}
\centerline{Commands marked with \dx{} require dired-x}

% trim this down to fit everything on one page
% \section{General}
% In dired, you can edit a list of the files in a directory (and optionally
% its subdirectories in the `ls -lR' format).

% Editing a directory means that you can visit, rename, copy, compress,
% load, byte-compile  files.  You can change files' attributes, run shell
% commands on files, or insert subdirectories into the edit buffer.  You can
% "flag" files for deletion or "mark" files for later commands, either one
% file at a time or by all files matching certain criteria (e.g., files that
% match a certain regexp).

% You move throughout the buffer using the usual cursor motion commands.
% Letters no longer insert themselves, but execute commands instead.  The
% digits (0-9) are prefix arguments.

% Most commands operate either on all marked files or on the current file if
% no files are marked.  Use a numeric prefix argument to operate on the next
% ARG files (or previous ARG if ARG $<$ 0).  Use the prefix argument `1' to
% operate on the current file only.  Prefix arguments override marks. Commands
% which run a sub-process on a group of files will display a list of files
% for which the sub-process failed.  Typing y will try to tell
% you what went wrong.

% When editing several directories in one buffer, each directory acts as a
% page, so C-x [ and C-x ] can be used to move between directories.

\section{Entering and Exiting Dired}

\key{run dired}{C-x d}
\key{dired the directory of the file you are editing}{C-x C-j \dx}
\key{quit dired}{q}

\section{Motion Commands}

\key{move up to previous line}{p}
\key{move down to next line}{n}
\key{move up to previous directory line}{<}
\key{move down to next directory line}{>}
\key{move to next marked file}{M-\}}
\key{move to previous marked file}{M-\{}
\key{move up to previous subdirectory}{M-C-p}
\key{move down to next subdirectory}{M-C-n}
\key{move to parent directory}{^}
\key{move to first child subdirectory}{M-C-d}

\section{Mouse Commands}
\metax{visit file}{Mouse_Button_2}
\metax{popup menu}{Control-Mouse_Button_3}

\section{Immediate Actions on Files}

\key{visit current file}{f}
\key{view current file}{v}
\key{visit current file in other window}{o}
\key{visit current file in other frame}{w}
\key{display current file}{C-u o}
\key{create a new subdirectory}{+}
\key{compare file at point with the one at mark}{=}

\section{Marking and Unmarking Files}

\key{mark a file or subdirectory for later commands}{m}
\key{unmark a file or all files of a subdirectory}{u}
\key{unmark all marked files in a buffer}{M-delete}
\key{mark files with a given extension}{* .}
\key{mark all directories}{* /}
\key{mark all symlinks}{* @}
\key{mark all executables}{* *}
\key{invert marking}{* t}
\key{mark all files in the current subdir}{* s}
\key{mark file names matching a regular expression}{* \%}
\key{change the marks to a different character}{* c}
\key{mark files for which Elisp expression returns t}{* ( \dx}

\section{Modifying the Dired Buffer}

\key{insert a subdirectory into this buffer}{i}
\key{remove marked files from the listing}{k}
\key{remove a subdir listing}{C-u k}
\key{re-read all directories (retains all marks)}{g}
\key{toggle sorting of current subdir by name/date}{s}
\key{edit ls switches}{C-u s}
\key{recover marks, hidden lines, and such}{C-_}
\key{hide all subdirectories}{M-\$}
\key{hide or unhide subdirectory}{\$}

\section{Commands on Files Marked or Specified by the Prefix}

\key{copy file(s)}{C}
\key{rename a file or move files to another directory}{R}
\key{change ownership of file(s)}{O}
\key{change the group of the file(s)}{G}
\key{change mode of file(s)}{M}
\key{print file(s)}{P}
\key{convert filename(s) to lower case}{\% l}
\key{convert filename(s) to upper case}{\% u}
\key{delete marked (as opposed to flagged) files}{X}
\key{uuencode or uudecode file(s)}{U}
\key{compress or uncompress file(s)}{Z}
\key{run info on file}{I \dx}
\key{make symbolic link(s)}{S}
\key{make relative symbolic links}{Y}
\key{make hard link(s)}{H}
\key{search files for a regular expression}{A}
\key{query replace regular expression}{Q}
\key{byte-compile files}{B}
\key{load files}{L}
\key{shell command on file(s)}{!}

\section{Flagging Files for Deletion}
\leftline{\bf Unmark commands remove delete flags}
\key{flag file for deletion}{d}
\key{backup and remove deletion flag}{delete}
\key{flag all backup files (file names ending in \~{})}{\~{}}
\key{flag all auto-save files}{\#}
\key{flag various intermediate files}{\&}
\key{flag numeric backups (ending in .\~{}1\~{}, .\~{}2\~{}, etc.)}{.}
\key{execute the deletions requested (flagged files)}{x}
\key{flag files matching a regular expression}{\% d}

\section{Regular Expression Commands}

\key{mark filenames matching a regular expression}{\% m}
\key{copy marked files by regexp}{\% C}
\key{rename marked files by regexp}{\% R}
\key{hardlink}{\% H}
\key{symlink}{\% S}
\key{symlink, with relative paths}{\% Y}
\key{mark for deletion}{\% d}

\section{Dired and Find}
\metax{dired file(s) whose name matches a pattern}{M-x find-name-dired}
\metax{dired file(s) that contain pattern}{M-x find-grep-dired}
\metax{dired file(s) based on \kbd{find} output}{M-x find-dired}

\section{Getting Help}

\key{dired help}{h}
\key{dired summary (short help) and error log}{?}

\copyrightnotice

\bye

% arch-tag: 483e7bb4-4576-4733-9fca-0eae1c861484
