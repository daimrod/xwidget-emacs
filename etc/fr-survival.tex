%&tex
%
% Title:  GNU Emacs Survival Card
% Author: Wlodek Bzyl <matwb@univ.gda.pl>
%
%**start of header

% User interface is `plain.tex' and macros described below
%
% \title{CARD TITLE}{for version 21}
% \section{NAME}
% optional paragraphs separated with \askip amount of vertical space
% \key{KEY-NAME} description of key or
% \mkey{M-x LONG-LISP-NAME} description of Elisp function
%
% \kbd{ARG} -- argument is typed literally

\def\plainfmtname{plain}
\ifx\fmtname\plainfmtname
\else
  \errmessage{This file requires `plain' format to be typeset correctly}
  \endinput
\fi

% Copyright (C) 2000, 2002, 2003, 2004, 2005 Free Software Foundation, Inc.

% This file is part of GNU Emacs.

% GNU Emacs is free software; you can redistribute it and/or modify
% it under the terms of the GNU General Public License as published by
% the Free Software Foundation; either version 2, or (at your option)
% any later version.

% GNU Emacs is distributed in the hope that it will be useful,
% but WITHOUT ANY WARRANTY; without even the implied warranty of
% MERCHANTABILITY or FITNESS FOR A PARTICULAR PURPOSE.  See the
% GNU General Public License for more details.

% You should have received a copy of the GNU General Public License
% along with GNU Emacs; see the file COPYING.  If not, write to
% the Free Software Foundation, Inc., 51 Franklin Street, Fifth Floor,
% Boston, MA 02110-1301, USA.

% Translated to french by \'Eric Jacoboni <jaco@teaser.fr> in November 2001

\def\versionnumber{1.0}
\def\year{2005}

\def\copyrightnotice{\penalty-1\vfill
  \vbox{\smallfont\baselineskip=0.8\baselineskip\raggedcenter
    Copyright \year\ Free Software Foundation, Inc.\break
    Version \versionnumber{} pour GNU Emacs 21, Avril 2000\break
    Auteur W{\l}odek Bzyl (matwb@univ.gda.pl)\break
    Traduction fran\c{c}aise \'Eric Jacoboni (jaco@teaser.fr)

    Vous pouvez faire et distribuer des copies de cette carte, pourvu
    que la note de copyright, ainsi que cette notice soient
    pr\'eserv\'ees sur toutes les copies.\par}}

\hsize 3.2in
\vsize 7.95in
\font\titlefont=cmss10 scaled 1200
\font\headingfont=cmss10
\font\smallfont=cmr6
\font\smallsy=cmsy6
\font\eightrm=cmr8
\font\eightbf=cmbx8
\font\eightit=cmti8
\font\eighttt=cmtt8
\font\eightmi=cmmi8
\font\eightsy=cmsy8
\font\eightss=cmss8
\textfont0=\eightrm
\textfont1=\eightmi
\textfont2=\eightsy
\def\rm{\eightrm} \rm
\def\bf{\eightbf}
\def\it{\eightit}
\def\tt{\eighttt}
\def\ss{\eightss}
\baselineskip=0.8\baselineskip

\newdimen\intercolumnskip % horizontal space between columns
\intercolumnskip=0.5in

% The TeXbook, p. 257
\let\lr=L \newbox\leftcolumn
\output={\if L\lr
    \global\setbox\leftcolumn\columnbox \global\let\lr=R
  \else
       \doubleformat \global\let\lr=L\fi}
\def\doubleformat{\shipout\vbox{\makeheadline
    \leftline{\box\leftcolumn\hskip\intercolumnskip\columnbox}
    \makefootline}
  \advancepageno}
\def\columnbox{\leftline{\pagebody}}

\def\newcolumn{\vfil\eject}

\def\bye{\par\vfil\supereject
  \if R\lr \null\vfil\eject\fi
  \end}

\outer\def\title#1#2{{\titlefont\centerline{#1}}\vskip 1ex plus 0.5ex
   \centerline{\ss#2}
   \vskip2\baselineskip}

\outer\def\section#1{\filbreak
  \bskip
  \leftline{\headingfont #1}
  \askip}
\def\bskip{\vskip 2.5ex plus 0.25ex }
\def\askip{\vskip 0.75ex plus 0.25ex}

\newdimen\defwidth \defwidth=0.25\hsize
\def\hang{\hangindent\defwidth}

\def\textindent#1{\noindent\llap{\hbox to \defwidth{\tt#1\hfil}}\ignorespaces}
\def\key{\par\hangafter=0\hang\textindent}

\def\mtextindent#1{\noindent\hbox{\tt#1\quad}\ignorespaces}
\def\mkey{\par\hangafter=1\hang\mtextindent}

\def\kbd#{\bgroup\tt \let\next= }

\newdimen\raggedstretch
\newskip\raggedparfill \raggedparfill=0pt plus 1fil
\def\nohyphens
   {\hyphenpenalty10000\exhyphenpenalty10000\pretolerance10000}
\def\raggedspaces
   {\spaceskip=0.3333em\relax
    \xspaceskip=0.5em\relax}
\def\raggedright
   {\raggedstretch=6em
    \nohyphens
    \rightskip=0pt plus \raggedstretch
    \raggedspaces
    \parfillskip=\raggedparfill
    \relax}
\def\raggedcenter
   {\raggedstretch=6em
    \nohyphens
    \rightskip=0pt plus \raggedstretch
    \leftskip=\rightskip
    \raggedspaces
    \parfillskip=0pt
    \relax}

\chardef\\=`\\

\raggedright
\nopagenumbers
\parindent 0pt
\interlinepenalty=10000
\hoffset -0.2in
%\voffset 0.2in

%**end of header


\title{Carte de survie pour GNU\ \ Emacs}{version 21}

Dans ce qui suit, \kbd{C-z} signifie qu'il faut appuyer sur la touche
`\kbd{z}' tout en maintenant la touche {\it Ctrl}\ \
press\'ee. \kbd{M-z} signifie qu'il faut appuyer sur la touche
`\kbd{z}' tout en maintenant la touche {\it Meta\/} (marqu\'ee {\it Alt\/}
sur certains claviers) ou apr\`es avoir press\'e la touche {\it Echap\/} key.

\section{Lancement de Emacs}

Pour lancer GNU Emacs, il suffit de taper son nom~: \kbd{emacs}.
Emacs divise son cadre en plusieurs parties~:
  une ligne de menu,
  une zone tampon contenant le texte \'edit\'e,
  une ligne de mode d\'ecrivant le tampon de la fen�tre au-dessus d'elle,
  et un mini-tampon/zone d'\'echo sur la derni\`ere ligne.
\askip
\key{C-x C-c} quitte Emacs
\key{C-x C-f} \'edite une fichier~; cette commande utilise le
mini-tampon pour lire le nom du fichier~; utilisez-la pour cr\'eer de
nouveaux fichiers en entrant le nom du fichier \`a cr\'eer
\key{C-x C-s} sauve le fichier
\key{C-x k} supprime un tampon
\key{C-g} dans la plupart des contextes~: annule, stoppe, avorte une
commande en cours d'ex\'ecution ou de saisie
\key{C-x u} annule

\section{D\'eplacements}

\key{C-l} place la ligne courante au centre de la fen\^etre
\key{C-x b} bascule dans un autre tampon
\key{M-<} va au d\'ebut du tampon
\key{M->} va \`a la fin du tampon
\key{M-x goto-line} va \`a la ligne indiqu\'ee

\section{Fen\^etres multiples}

\key{C-x 0} \^ote la fen\^etre courante de l'affichage
\key{C-x 1} ne conserve que la fen\^etre active
\key{C-x 2} divise la fen\^etre dans le sens de la hauteur
\key{C-x 3} divise la fen\^etre dans le sens de la largeur
\key{C-x o} va dans une autre fen\^etre

\section{R\'egions}

Emacs d\'efinit une 'r\'egion' comme l'espace entre la {\it marque\/} et
le {\it point}. On positionne une marque avec \kbd{C-{\it espace}}.
Le point est la position courante du curseur.
\askip
\key{M-h} marque le paragraphe entier
\key{C-x h} marque le tampon entier

\section{Suppression et copie}

\key{C-w} supprime la r\'egion
\key{M-w} copie la r\'egion dans le 'kill-ring'
\key{C-k} supprime du curseur jusqu'\`a la fin de la ligne
\key{M-DEL} supprime le mot
\key{C-y} restaure la derni\`ere suppression (la combinaison \kbd{C-w
  C-y} sert \`a se d\'eplacer dans le texte)
\key{M-y} remplace la derni\`ere restauration avec la suppression pr\'ec\'edente

\section{Recherche}

\key{C-s} recherche une cha\^\i{}ne
\key{C-r} recherche une cha\^\i{}ne vers l'arri\`ere
\key{RET} quitte la recherche
\key{M-C-s} recherche par expression rationnelle
\key{M-C-r} recherche par expression rationnelle vers l'arri\`ere
\askip
R\'ep\'etez \kbd{C-s} ou \kbd{C-r} pour renouveler une recherche dans une
des deux directions.

\section{Marqueurs}

Les fichiers de tableaux de marqueurs enregistrent les emplacements des
d\'efinitions de fonctions ou de proc\'edures, des variables globales, des
types de donn\'ees et de tout ce qui peut \^etre pratique. Pour cr\'eer un
tel fichier, tapez `{\tt etags} {\it fichier\_entr\'ee}' \`a l'invite du shell.
\askip
\key{M-.} trouve une d\'efinition
\key{C-u M-.} trouve l'occurrence suivante de la d\'efinition
\key{M-*} revient o\`u \kbd{M-.} a \'et\'e appel\'e pour la derni\`ere fois
\mkey{M-x tags-query-replace} lance query-replace sur tous les
fichiers enregistr\'es dans le tableau des marqueurs
\key{M-,} continue la derni\`ere recherche de marqueurs ou le dernier
query-replace

\section{Compilation}

\key{M-x compile} compile le code situ\'e dans la fen\^etre active
\key{C-c C-c} va \`a l'erreur de compilation suivante, lorsque l'on est
dans la fen\^etre de compilation, ou
\key{C-x `} lorsque l'on est dans la fen\^etre du code source

\section{Dired, l'\'editeur de r\'epertoires}

\key{C-x d}  appelle Dired
\key{d} marque ce fichier pour une suppression
\key{\~{}} marque tous les fichiers de sauvegarde pour leur suppression
\key{u} supprime la marque de suppression
\key{x} supprime les fichiers marqu\'es pour suppression
\key{C} copie le fichier
\key{g} met \`a jour le tampon de Dired
\key{f} visite le fichier d\'ecrit sur la ligne courante
\key{s} bascule entre ordre alphab\'etique et ordre date/heure

\section{Lecture et envoi de courrier}

\key{M-x rmail} d\'emarre la lecture du courrier
\key{q} quitte la lecture du courrier
\key{h} montre les ent\^etes
\key{d} marque le message courant pour suppression
\key{x} supprime tous les messages marqu\'es pour suppression

\key{C-x m} d\'ebute la composition d'un message
\key{C-c C-c} envoie le message et bascule dans un autre tampon
\key{C-c C-f C-c} va \`a l'ent\^ete `CC', en cr\'ee un s'il n'existe pas

\section{Divers}

\key{M-q} formate le paragraphe
\key{M-/} expanse dynamiquement le mot pr\'ec\'edent
\key{C-z} iconifie (suspend) Emacs lorsqu'il s'ex\'ecute sous X ou
  sous un shell, respectivement
\mkey{M-x revert-buffer} remplace le texte en cours d'\'edition par le
texte du fichier sur disque

\section{Remplacement interactif}

\key{M-\%} cherche et remplace interactivement
\key{M-C-\%} utilise les expressions rationnelles
\askip
Les r\'eponses correctes dans le mode query-replace sont :
\askip
\key{SPC} remplace celui-ci, passe au suivant
\key{,} remplace ce celui-ci, pas de d\'eplacement
\key{DEL} passe au suivant sans remplacer celui-ci
\key{!} remplace toutes les occurrences suivantes
\key{\^{}} revient \`a l'occurrence pr\'ec\'edente
\key{RET} quitte query-replace
\key{C-r} entre en \'edition r\'ecursive (\kbd{M-C-c} pour en sortir)

\section{Expressions rationnelles}

\key{. {\rm(point)}} n'importe quel caract\`ere unique, sauf la fin de ligne
\key{*} z\'ero r\'ep\'etition ou plus
\key{+} une r\'ep\'etition ou plus
\key{?} z\'ero ou une r\'ep\'etition
\key{[$\ldots$]} repr\'esente une classe de caract\`eres
\key{[\^{}$\ldots$]} compl\'emente la classe

\key{\\{\it c}} prot\`ege les caract\`eres qui, sinon, auraient une
  signification sp\'eciale dans les expressions rationnelles.

\key{$\ldots$\\|$\ldots$\\|$\ldots$} correspond \`a une
alternative (``ou'') .
\key{\\( $\ldots$ \\)} groupe une suite d'\'el\'ements de motif pour
former un \'el\'ement unique.
\key{\\{\it n}} le m\^eme texte que le {\it n\/}i\`eme groupe.

\key{\^{}} correspond au d\'ebut de ligne
\key{\$} correspond \`a la fin de ligne

\key{\\w} correspond \`a un caract\`ere de mot
\key{\\W} correspond \`a ce qui n'est pas un caract\`ere mot
\key{\\<} correspond au d\'ebut d'un mot
\key{\\>} correspond \`a la fin d'un mot
\key{\\b} correspond \`a une coupure de mot
\key{\\B} correspond \`a ce qui n'est pas une une coupure de mot

\section{Registres}

\key{C-x r s} sauve la r\'egion dans un registre
\key{C-x r i} ins\`ere le contenu d'un registre dans le tampon

\key{C-x r SPC} sauve la valeur du point dans un registre
\key{C-x r j} va au point sauvegard\'e dans un registre

\section{Rectangles}

\key{C-x r r} copie le rectangle dans un registre
\key{C-x r k} supprime le rectangle
\key{C-x r y} restaure le rectangle
\key{C-x r t} pr\'efixe chaque ligne d'une cha\^\i{}ne

\key{C-x r o} ouvre un rectangle en d\'ecalant le texte vers la droite
\key{C-x r c} vide le rectangle

\section{Shells}

\key{M-x shell} lance un shell dans Emacs
\key{M-!} ex\'ecute une commande dans un shell
\key{M-|} lance une commande shell sur la r\'egion
\key{C-u M-|} filtre la r\'egion via une commande shell

\section{V\'erification orthographique}

\key{M-\$} v\'erifie l'orthographe du mot sous le curseur
\mkey{M-x ispell-region} v\'erifie l'orthographe de tous les mots de la r\'egion
\mkey{M-x ispell-buffer} v\'erifie l'orthographe de tout le tampon

\section{Jeux de caract\`eres internationaux}

\key{C-x RET C-\\} s\'electionne et active une m\'ethode d'entr\'ee pour le
  tampon courant
\key{C-\\} active ou d\'esactive la m\'ethode d'entr\'ee
\mkey{M-x list-input-methods} affiche toutes les m\'ethodes d'entr\'ee
\mkey{M-x set-language-environment} pr\'ecise la langue principale

\key{C-x RET c} fixe le syst\`eme de codage pour la commande suivante
\mkey{M-x find-file-literally} visite un fichier sans aucune conversion

\mkey{M-x list-coding-systems} affiche tous les syst\`emes de codage
\mkey{M-x prefer-coding-system} choisit le syst\`eme de codage pr\'ef\'er\'e

\section{Macros clavier}

\key{C-x (} lance la d\'efinition d'une macro clavier
\key{C-x )} termine la d\'efinition d'une macro clavier
\key{C-x e} ex\'ecute la derni\`ere macro clavier d\'efinie
\key{C-u C-x (} ajoute \`a la derni\`ere macro clavier
\mkey{M-x name-last-kbd-macro} donne un nom \`a la derni\`ere macro clavier

\section{Personnalisation simple}

\key{M-x customize} personnalise les variables et les fontes

\section{Obtenir de l'aide}

Emacs effectue pour vous la compl\'etion des commandes. En faisant \kbd{M-x}
{\it tab\/} ou {\it espace\/}, vous obtiendrez une liste des commandes
Emacs.
\askip
\key{C-h} aide d'Emacs
\key{C-h t} lance le didacticiel d'Emacs
\key{C-h i} lance Info, le navigateur de documentations
\key{C-h a} affiche les commandes correspondant \`a une cha\^\i{}ne (apropos)
\key{C-h k} affiche la documentation de la fonction appel\'ee par la
combinaison de touches
\askip
Emacs utilise diff\'erents {\it modes}, chacun d'eux personnalisant
Emacs pour l'\'edition de textes de diff\'erents types. La ligne de mode
contient entre parenth\`eses le nom des modes en cours.
\askip
\key{C-h m} affiche les informations sp\'ecifiques au mode en cours

\copyrightnotice

\bye

% Local variables:
% compile-command: "tex survival"
% End:

% arch-tag: 2fb4e93f-8bfa-4ab4-bc6d-b475131d766a
